\section{\underline{Central Potentials}}
Let us know look at the special case where we have spinless particles moving in the presence of a central potential. These central potentials will only depend on the magnitude of $\vec{r}$ (i.e., they are spherically symmetric) and will be denoted by $V(r)$. Introduction of intrinsic spin is straight-forward by is an additional complication which does not give any additional physical insight into the problem -- So we will delay its introduction until later. Furthermore we will only look
at the bound states at this time. 

For a single particle moving in a central potential, the time independent wave equation is
$$H\psi(\vec{r}) = \left( {p^2\over2m}+V(r)\right) \psi(\vec{r}) = E\psi(\vec{r}) \mbox{~~~ or  }$$
$$\left( -{\hbar^2\over2m}\nabla^2 + V(r) - E \right) \psi(\vec{r}) = 0.$$
If we stay in the Cartesian coordinate system, one cannot go much further until $V(r)$ is specified. But because of the spherical symmetry of $V(r)$, it is much more desirale to approach this problem in spherical coordinates. To do this, we
Introduce the orbital angular momentum operator $\ds \vec{L} = \vec{r} \times \vec{p}$. In component form, we have $\ds L_x = y p_z - zp_y$ (cyclic). By cyclic, we mean that this formula holds for $x\goes y \goes z \goes x$. Also we have
$[x,p_x] = i\hbar$ (cyclic). With these relations, it is easy to show that 
$$[L_x, L_y] = i\hbar L_z\mbox{~~~(cyclic)}.$$
Symbolically, we can write the three equations implied above as 
$$\vec{L}\times \vec{L} = i\hbar \vec{L}.$$
Using the above commutation relations for the $L_i$'s, one can easily show with 
$$L^2 = L_x^2 + L_y^2 + L_z^2$$ that 
$$[L_x, L^2] = [L_y,L^2] = [L_z,L^2] = 0.$$
In spherical coordinates, one has 
\begin{eqnarray*}
L_x &=& ~i\hbar \left[ \sin\phi {\partial \over \partial \theta} + \cot\theta\cos\phi{\partial \over\partial\phi}\right],\\
L_y &=& -i\hbar \left[ \cos\phi {\partial \over \partial \theta} - \cot\theta\sin\phi{\partial \over\partial\phi}\right],\\
L_z &=& -i\hbar {\partial \over\partial\phi},
\end{eqnarray*}
and 
$$L^2 = -\hbar^2\left[ {1\over\sin\theta} {\partial\over\theta} \sin\theta {\partial\over\partial\theta} + {1\over\sin^2\theta} {\partial^2\over\partial\phi^2} \right].$$

With this expression for $L^2$, one can easily show that in spherical coordinates, 
\begin{eqnarray*}
H = {p^2\over2m} + V(r) &=& -{\hbar^2\over2m}{1\over r}{\partial^2\over \partial r^2} r + {L^2\over2mr^2} + V(r) \\
                                       &=& -{\hbar^2\over2m}{1\over r}{\partial\over\partial r} r^2{\partial\over\partial r} + {L^2\over2mr^2} + V(r)
\end{eqnarray*}
So we see that all the $\theta$ and $\phi$ dependence of our Hamiltonian operator (or of $\nabla^2$) is contained in the operator $L^2$. 

The standard approach to solving 
$$H\psi(\vec{r}) = E\psi(\vec{r})$$ is to assume separable solutions, i.e, 
$$\psi(\vec{r}) = R(r)\Theta(\theta)\Phi(\phi) = R(r)f(\theta, \phi).$$
I will assume that you have already done this in an elementary course in quantum mechanics. What you did and/or found was that one could 
solve for $f(\theta,\phi)$ which was a solution to the differential equation
$$L^2 f(\theta,\phi) = c f(\theta,\phi).$$
The solutions to this differential equation are the spherical harmonics $Y_{lm}(\theta,\phi)$ which consists of products of associated Legendre 
polynomials and $\ds e^{\pm i m \phi}$'s. The solutions, $Y_{lm}$ have the properties of            
\begin{eqnarray*}
L^2 \Ylm &=& l(l+1)\hbar^2\,\Ylm\mbox{  and}\\
L_z\Ylm  &=& m\hbar\, \Ylm 
\end{eqnarray*}                            
So the $Y_{lm}$'s are simultaneous eigenfunctions of $L^2$ and $L_z$. -- This is okay as $L^2$ and $L_z$ commute. We could have easily looked for 
simultaneous eigenfunctions of $L^2$ and $L_x$ or of $L^2$ and $L_y$. -- But we couldn't look for simultaneous eigenfunctions of $L_y$ and $L_z$ as they don't commute, etc. for $L_x$ and $L_z$ or $L_x$ and $L_y$. 
The eigenfunctions of $L^2$ and $L_x$ or of $L^2$ and $_y$ can always be expanded as $\ds \sum_m\, a_m Y_{lm}$ and these eigenfunctions will still be eigenfunctions of $L^2$. However, it is simpler and conventional
to use the eigenfunctions of $L^2$ and $L_z$, viz., the $Y_{lm}$'s. 

In quantum mechanics, we know that all eigenfunctions are undefined up to an arbitrary phase, Recent quantum mechanical texts usually adopt what is known as the ``Condon and Shortley'' phases for the $Y_{lm}$'s and we will 
do this also. A word of caution however, all texts and/or articles do not necessarily do so -- see Pauling \& Wilson, 1st edition of Schiff. The phases when we adopt fo the $Y_{lm}$'s are
\begin{eqnarray*}
Y^*_{l,m}(\theta,\phi) &=& (-1)^m\, Y_{l,-m}(\theta,\phi) \mbox{ ~~~~ with} \\
\Ylm &=& (-1)^m \sqrt{(2l+1)(l-|m)! \over 4\pi(l+|m|)!}  \, P_l^{|m|}(\cos\theta)e^{im\phi}.
\end{eqnarray*}
The $Y^*_{lm}$'s and $\Ylm$'s are orthonormal such that 
\[ \int\, Y^*_{lm}(\theta,\phi) Y_{l'm'}(\theta,\phi)\, \sin\theta\, d\theta d\phi  \ = \int \, Y^*_{l'm'}(\Omega)Y_{l'm'}(\Omega)\, d\Omega = \delta_{ll'} \delta_{mm'} \]
where $d\Omega \equiv \sin\theta \, d\theta d\phi $ = differential element of solid angle. (See Schiff pp. 76-83 for more details).

With this digression dealing with the spherical harmonics, $\Ylm$, which by the way form a complete set in $\theta, \phi$ space, let us now return to our central potential problem with 
\[ H = {p^2\over2m} = -{\hbar^2\over2m} {1\over r} {\partial^2 \over \partial r^2} + {L^2\over2mr^2} + V(r). \]
Instead of solving $\ds H \psi(\vec{r}) = E\psi(\vec{r})$ using separable solutions, we note that the $\Ylm$'s form a complete set and so we expand $\psi(\vec{r})$ in terms of them, viz.,
\[ \psi(\vec{r}) = \sum_{l',m'} \, R_{l'm'} Y_{l'm'}(\theta, \phi). \]
Since $L^2 Y_{l'm'} = l'(l'+1)\hbar^2 Y_{\l'm'}$, our differential 
equation becomes
\[ \sum_{l'm'}\, \left[ {1\over r} {\partial^2\over \partial r^2} r - {l'(l'+1)\over r^2} + {2m\over\hbar^2}(E - V(r))\right] R_{l'm'}Y_{l'm'} = 0 \]
Multiplying by $Y^*_{lm}(\theta,\phi)$ and integrating over $\theta$ and $\phi$, the orthogonality of the $\Ylm$'s give us
\[ \left[ {1\over r} {\partial^2\over \partial r^2} r  -{l(l+1)\over r^2} + {2m\over\hbar^2}(E - V(r))\right] R_{lm}(r) = 0 \] which is the differential equation which we must solve for $R_{lm}(r)$, the radial solutions. 

For a central potential, $V(r)$, the ``magnetic quantum number'', $m$, does not enter -- the $m$ in $\ds {2m\over\hbar^2}$ is the mass -- and so we can drop the subscript $m$ on $R_{lm}(r)$ and write it simple as $R_l(r)$.

In the more general case where $V(\vec{r})$ is not spherically symmetric, $R_{lm}(r)$ will depend on $m$ and we will have to retain it in our solution.

If our particles have intrinsic spin, $\vec(s)$, the only change will be to replace $\Ylm$ above by $\Ylm \chi_{sm_s}$ where $X_{sm_s}$ are eigenfunctions of the spin operator. -- We will do this in more detail later. 

In order to proceed further, we need to now the form of $V(r)$. We will now look at three rather common cases -- The square well, the hydrogen atom, and the three dimensional harmonic oscillator. I will assume that you 
have already done these three cases before. I will only summarize them and let you fill in the details.

\subsection{\underline{Square Well (Bound State}}
For a square well, we have 
\begin{eqnarray*}
V(r) &=& - V_0\mbox{~~~ for ~} 0\le r \le a \\
V(r) &=& 0 \mbox{~~~~ for ~} r > a
\end{eqnarray*}
We define 
\begin{eqnarray*}
K &=&  \sqrt{ {2m\over\hbar^2} )V_0 - |E| } \mbox{ and } \rho = Kr \mbox{ for } r\le a \\
k &=& \sqrt{{2m|E|\over\hbar^2}}\mbox{  and  } \rho = ikr \mbox{ for } r>a
\end{eqnarray*}
For $r\le a$ and $\rho = Ka$, our differential equation becomes 
\[ {d^2R_l(\rho)\over d\rho^2} + {2\over\rho} {dR_l(\rho)\over d\rho} + \left[ 1- {l(l+1)\over\rho^2}   \right]\, R_l(\rho) = 0. \]
The solutions to this differential equation are the spherical Bessel functions $j_l(\rho)$ and $n_l(\rho)$ where 
\begin{eqnarray*}
k_l(\rho) &=& \sqrt{\pi\over2\rho} J_{l+\half} (\rho) \mbox{ and is regular at the origin and }\\
n_l(\rho) &=& (-1)^{l+1} \sqrt{\pi\over2\rho} J_{-(l+\half)}(\rho) \mbox{ and is irregular at the origin}.
\end{eqnarray*}

Since $n_l(\rho)$ is irregular at the origin, it must be excluded in the solution $R_l(\rho)$ for $r\le a$. Then $R_l(\rho) \propto j_l(\rho)$ for $r \le a$. 

The asymptotic properties of $j_l$ and $n_l$ which we will use quite often are
\begin{eqnarray*}
\lim_{\rho\goes 0} j_l(\rho) &\goes& {\rho^l\over (2l+1)!!}, ~~~~~~ \lim_{\rho\goes\infty} j_l(\rho) \goes {\cos[ \rho-(l+1){\pi\over2}]\over \rho}, \\
\lim_{\rho\goes 0} n_l(\rho) &\goes& {(2l-1)!!\over\rho^{l+1}}, ~~~~~~ \lim_{\rho\goes\infty} n_l(\rho) \goes {\sin[ \rho-(l+1){\pi\over2}]\over \rho}.
\end{eqnarray*}
where $(2l+1)!! \equiv (2l+1)(2l-1)(2l-3)\hdots5\cdot3\cdot1$.

In the ``exterior'' region where $r\ge a$, we have the same differential equation for $R_l(\rho)$ with $\rho = i k r$. Since the origin is now excluded we can expect that $R_l(\rho)$ will be in general a linear combination of $j_l$ and $n_l$. 
Normally it is more convenient to use instead the spherical Hankel functions $h^{(1)}_l(\rho)$ and $h^{(2)}(\rho)$ where 
\begin{eqnarray*}
h^{(1)}_l(\rho) &\equiv& j_l(\rho) + i  n_l(\rho) \mbox{  ~~ and }\\
h^{(2)}_l(\rho) &\equiv& j_l(\rho) - i  n_l(\rho),\\
\end{eqnarray*}
which have asymptotic behaviors of 
\begin{eqnarray*}
\lim_{\rho\goes\infty} h^{(1)}_l(\rho) &\goes& {e^{i[\rho-(l+1) {\pi\over2} ]} \over \rho} \mbox{  ~~ and }\\
\lim_{\rho\goes\infty} h^{(2)}_l(\rho) &\goes& {e^{-i[\rho-(l+1){\pi\over2} ]} \over \rho}.
\end{eqnarray*}

For our case of a square well, $\rho = -ikr$ and we see that the Hankel functions will describe increasing and decreasing exponentials in the asymptotic region where $rho \goes \infty$. Clearly, the increasing exponential will not be 
allowed for the bound states. 

Furthermore, our solutions, and their derivatives must be continuous across the ``boundary'' at $r = a$. Not every value of $E$ will satisfy this ``matching requirement''. As a result, only certain discrete values of $E$ will be 
allowed for bound states. Practically all quantum mechanics texts work out these solutions and I refer you to them for details. 

For an infinitely deep square well, the energy levels look somewhat like those shown at the right
\begin{figure}[h]
\centering
%\includegraphics[scale=0.5]{graph_a}
\caption{Infinite Square Well Energy Eigenvalues}
\label{fig:InfiniteSquareWell}
\end{figure}
Instead of labeling of the levels by $l = 0,1,2,3,\hdots$, one uses $s, p, d, f, g, h, i, \hdots$ respectively. The lowest $s$ levels is the $1s$ level, the next is the $2s$ level, etc.

This is one of the two most common labelling schemes. Another labeling scheme will be used for the Hydrogen atom which we will discuss next. 

\subsection{\underline{Hydrogen Atom (without spin)}}
Now we have a two-body problem but we will show that it reduces to a one-body problem. Our Hamiltonian is 
\[ H = {p_1^2\over2m_1} + {p_2^2\over2m_2} + V(|\vec{r_1} - \vec{r_2})|) \equiv E_{total} \] where one particle is an electron, the other is a proton, and
$\ds V(|\vec{r_1} - \vec{r_2})|)$ is the Coulomb potential between them. We now transform to the relative and center of mass coordinate using 
\[ \vec{r} = \vec{r_1} - \vec{r_2}, \quad \vec{R} = {m_1 \vec{r_1} + m_2\vec{r_2}\over m_1 + m_2}, \quad M = m_1 + m_2, \mbox{  and} \]
\[ m = {m_1m_2\over m_1 + m_2} ~~= \mbox{  reduced mass}.\]
This transformation gives 
\[ H = {P^2\over2M} + {p^2\over2m} + V(r), \mbox{~~where~~} \vec{P} = M\dot{\vec{R}}, \mbox{ ~ and ~} \vec{p} = m{\dot\vec{r}}. \]
We see that the Hamiltonian separates into two parts and so we can seek a separable solution 
$$\psi(\vec{r_1},\vec{r_r}) = \psi_{CM}(\vec{R}) \psi(\vec{r}).$$ 
The differential equation for $\psi_{CM}(\vec{R})$ is
$${P^2\over2M} \psi_{CM}(\vec{R}) = E_{CM} \psi_{CM}(\vec{R})$$ where $\ds E_{total} \equiv E_{CM} + E$.




