\section{\underline{Central Potentials}}
Let us know look at the special case where we have spinless particles moving in the presence of a central potential. These central potentials will only depend on the magnitude of $\vec{r}$ (i.e., they are spherically symmetric) and will be denoted by $V(r)$. Introduction of intrinsic spin is straight-forward by is an additional complication which does not give any additional physical insight into the problem -- So we will delay its introduction until later. Furthermore we will only look
at the bound states at this time. 

For a single particle moving in a central potential, the time independent wave equation is
$$H\psi(\vec{r}) = \left( {p^2\over2m}+V(r)\right) \psi(\vec{r}) = E\psi(\vec{r}) \mbox{~~~ or  }$$
$$\left( -{\hbar^2\over2m}\nabla^2 + V(r) - E \right) \psi(\vec{r}) = 0.$$
If we stay in the Cartesian coordinate system, one cannot go much further until $V(r)$ is specified. But because of the spherical symmetry of $V(r)$, it is much more desirale to approach this problem in spherical coordinates. To do this, we
Introduce the orbital angular momentum operator $\ds \vec{L} = \vec{r} \times \vec{p}$. In component form, we have $\ds L_x = y p_z - zp_y$ (cyclic). By cyclic, we mean that this formula holds for $x\goes y \goes z \goes x$. Also we have
$[x,p_x] = i\hbar$ (cyclic). With these relations, it is easy to show that 
$$[L_x, L_y] = i\hbar L_z\mbox{~~~(cyclic)}.$$
Symbolically, we can write the three equations implied above as 
$$\vec{L}\times \vec{L} = i\hbar \vec{L}.$$
Using the above commutation relations for the $L_i$'s, one can easily show with 
$$L^2 = L_x^2 + L_y^2 + L_z^2$$ that 
$$[L_x, L^2] = [L_y,L^2] = [L_z,L^2] = 0.$$
In spherical coordinates, one has 
\begin{eqnarray*}
L_x &=& ~i\hbar \left[ \sin\phi {\partial \over \partial \theta} + \cot\theta\cos\phi{\partial \over\partial\phi}\right],\\
L_y &=& -i\hbar \left[ \cos\phi {\partial \over \partial \theta} - \cot\theta\sin\phi{\partial \over\partial\phi}\right],\\
L_z &=& -i\hbar {\partial \over\partial\phi},
\end{eqnarray*}
and 
$$L^2 = -\hbar^2\left[ {1\over\sin\theta} {\partial\over\theta} \sin\theta {\partial\over\partial\theta} + {1\over\sin^2\theta} {\partial^2\over\partial\phi^2} \right].$$

With this expression for $L^2$, one can easily show that in spherical coordinates, 
\begin{eqnarray*}
H = {p^2\over2m} + V(r) &=& -{\hbar^2\over2m}{1\over r}{\partial^2\over \partial r^2} r + {L^2\over2mr^2} + V(r) \\
                                       &=& -{\hbar^2\over2m}{1\over r}{\partial\over\partial r} r^2{\partial\over\partial r} + {L^2\over2mr^2} + V(r)
\end{eqnarray*}
So we see that all the $\theta$ and $\phi$ dependence of our Hamiltonian operator (or of $\nabla^2$) is contained in the operator $L^2$. 

The standard approach to solving 
$$H\psi(\vec{r}) = E\psi(\vec{r})$$ is to assume separable solutions, i.e, 
$$\psi(\vec{r}) = R(r)\Theta(\theta)\Phi(\phi) = R(r)f(\theta, \phi).$$
I will assume that you have already done this in an elementary course in quantum mechanics. What you did and/or found was that one could 
solve for $f(\theta,\phi)$ which was a solution to the differential equation
$$L^2 f(\theta,\phi) = c f(\theta,\phi).$$
The solutions to this differential equation are the spherical harmonics $Y_{lm}(\theta,\phi)$ which consists of products of associated Legendre 
polynomials and $\ds e^{\pm i m \phi}$'s. The solutions, $Y_{lm}$ have the properties of            
\begin{eqnarray*}
L^2 \Ylm &=& l(l+1)\hbar^2\,\Ylm\mbox{  and}\\
L_z\Ylm  &=& m\hbar\, \Ylm 
\end{eqnarray*}                            
So the $Y_{lm}$'s are simultaneous eigenfunctions of $L^2$ and $L_z$. -- This is okay as $L^2$ and $L_z$ commute. We could have easily looked for 
simultaneous eigenfunctions of $L^2$ and $L_x$ or of $L^2$ and $L_y$. -- But we couldn't look for simultaneous eigenfunctions of $L_y$ and $L_z$ as they don't commute, etc. for $L_x$ and $L_z$ or $L_x$ and $L_y$. 
The eigenfunctions of $L^2$ and $L_x$ or of $L^2$ and $_y$ can always be expanded as $\ds \sum_m\, a_m Y_{lm}$ and these eigenfunctions will still be eigenfunctions of $L^2$. However, it is simpler and conventional
to use the eigenfunctions of $L^2$ and $L_z$, viz., the $Y_{lm}$'s. 

In quantum mechanics, we know that all eigenfunctions are undefined up to an arbitrary phase, Recent quantum mechanical texts usually adopt what is known as the ``Condon and Shortley'' phases for the $Y_{lm}$'s and we will 
do this also. A word of caution however, all texts and/or articles do not necessarily do so -- see Pauling \& Wilson, 1st edition of Schiff. The phases when we adopt fo the $Y_{lm}$'s are
\begin{eqnarray*}
Y^*_{l,m}(\theta,\phi) &=& (-1)^m\, Y_{l,-m}(\theta,\phi) \mbox{ ~~~~ with} \\
\Ylm &=& (-1)^m \sqrt{(2l+1)(l-|m)! \over 4\pi(l+|m|)!}  \, P_l^{|m|}(\cos\theta)e^{im\phi}.
\end{eqnarray*}
The $Y^*_{lm}$'s and $\Ylm$'s are orthonormal such that 
\[ \int\, Y^*_{lm}(\theta,\phi) Y_{l'm'}(\theta,\phi)\, \sin\theta\, d\theta d\phi  \ = \int \, Y^*_{l'm'}(\Omega)Y_{l'm'}(\Omega)\, d\Omega = \delta_{ll'} \delta_{mm'} \]
where $d\Omega \equiv \sin\theta \, d\theta d\phi $ = differential element of solid angle. (See Schiff pp. 76-83 for more details).

With this digression dealing with the spherical harmonics, $\Ylm$, which by the way form a complete set in $\theta, \phi$ space, let us now return to our central potential problem with 
\[ H = {p^2\over2m} = -{\hbar^2\over2m} {1\over r} {\partial^2 \over \partial r^2} + {L^2\over2mr^2} + V(r). \]
Instead of solving $\ds H \psi(\vec{r}) = E\psi(\vec{r})$ using separable solutions, we note that the $\Ylm$'s form a complete set and so we expand $\psi(\vec{r})$ in terms of them, viz.,
\[ \psi(\vec{r}) = \sum_{l',m'} \, R_{l'm'} Y_{l'm'}(\theta, \phi). \]
Since $L^2 Y_{l'm'} = l'(l'+1)\hbar^2 Y_{\l'm'}$, our differential 