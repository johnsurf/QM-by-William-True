%   11/11/92 211121724  MEMBER NAME  REPORT1  (TEXFILES) M  TEX
%format: latex    %hyphen: English
%\documentstyle[11pt,leqno,epsfig]{note}
%\documentstyle[11pt,epsfig]{note}
%\documentstyle[11pt,epsfig,pstimes,psmath]{note}
%\documentstyle[epsf]{article}
%\documentstyle{article}
%\usepackage{amsmath}
\documentclass[epsf]{article}

\font\twelverm=cmr12
\font\tenbf=cmbx10
\font\tenrm=cmr10
\font\tenit=cmti10
\font\elevenbf=cmbx10 scaled\magstep 1
\font\elevenrm=cmr10 scaled\magstep 1
\font\elevenit=cmti10 scaled\magstep 1
\font\ninebf=cmbx9
\font\ninerm=cmr9
\font\nineit=cmti9
\font\eightbf=cmbx8
\font\eightrm=cmr8
\font\eightit=cmti8
\font\sevenrm=cmr7

\usepackage{amsmath,amssymb,latexsym} 
%\usepackage{amsmath,amssymb,latexsym,hyperref} 
%\usepackage{easybmat}
%\usepackage{multirow,bigdelim}

%
%  `pstimes.sty' sets \rm to be Times, and `psmath.sty' sets the math
%  in Times as well.
%
\textwidth = 6.5truein
\textheight = 8.5truein
\oddsidemargin=0.2truein
\evensidemargin=0.2truein
\topmargin=0.50truein
%\notename{UCD/IIRPA} \notenumber{92-24}
\begin{document}
\null
%\hbox to \hsize{\hfill UCD/IIRPA~92-24}
\begin{center}{{\tenbf TRUE'S NOTES ON QUANTUM MECHANICS}}
		\vglue 1.0cm
 	       {\tenrm  WILLIAM \,\,TRUE\\}
	        \baselineskip=13pt
	       {\tenit Physics Department\\
	            University of California, Davis\\}
\baselineskip=12pt
               {\tenit Davis, CA 95616, USA\\}
\end{center}
%
%\newmathalphabet{\oldcal}
%\addtoversion{times}{\oldcal}{cmsy}{m}{n}
\newcommand{\unit}[1]{\mbox{\it #1}}
\newcommand{\subs}[1]{\mbox{\scriptsize\it #1}}
\def\Q2{$Q^2$}
\def\Wftn{$\Psi(\vec{r},t)$}
\def\IntPsiSq{|\Psi|^2 d\vec{r}}
\def\sgp{$\sigma_{\gamma p}\:$}
\mathchardef\Lcur="324C
\def\crfive{\cr\noalign{\vskip 5pt}}
\def\crten{\cr\noalign{\vskip 10pt}}
\def\cofac{\hbox{cofactor}}
\def\gij{g_{ij}}
\def\gijinv{g^{ij}}
\def\half{1\over2}
\def\gab{g_{\alpha\beta}}
\def\gabinv{g^{\alpha\beta}}
\def\gmunu{g_{\mu\nu}}
\def\gmunuinv{g^{\mu\nu}}
\def\gmu{\gamma^\mu}
\def\gmudag{\gamma^{\mu\dag}}
\def\gnu{\gamma^\nu}
\def\gmuc{\gamma_\mu}
\def\gnuc{\gamma_\nu}
\def\gzero{\gamma^0}
\def\gone{\gamma^1}
\def\gtwo{\gamma^2}
\def\gthree{\gamma^3}
\def\gfive{\gamma_5}
\def\gfive{\gamma_5}
\def\gzeroc{\gamma_0}
\def\gonec{\gamma_1}
\def\gtwoc{\gamma_2}
\def\gthreec{\gamma_3}
\def\qsq{{Q^2\over4E_1^2}}
\def\qr2{{Q^2\over2}}
\def\q2{{Q^2}}
\def\hy{\hat{y}}
\def\be{\begin{equation}}
\def\ee{\end{equation}}
\newcommand{\bra}[1]{\langle #1|}
\newcommand{\ket}[1]{|#1\rangle}
\newcommand{\braket}[2]{\langle #1|#2\rangle}
\newcommand{\expect}[1]{\langle #1\rangle}

%\maketitle

%some definitions
%\setlength{\oddsidemargin}{0cm}
%\setlength{\textwidth}{17cm}
%\setlength{\topmargin}{-3cm}
%\setlength{\textheight}{25cm}
%\setlength{\unitlength}{1mm}
%\hsize=6.0 true in
%\vsize=8.40 true in
%\pagestyle{plain}
\parindent = 0.cm
\vglue 0.3cm
\rightskip=3pc
\leftskip=3pc
\twelverm
\noindent

\baselineskip 16pt

\tableofcontents
\newpage

\underline{Some Quantum Mechanics Texts}

\begin{description}
\item[] Schiff  --- Quantum Mechanics (3rd. Edition) ---Text
\item[] Messiah --- Quantum Mechanics (Vol. I and II) 
\item[] Davydov --- Quantum Mechanics 
\item[] Baym    --- Lectures on Quantum Mechanics
\item[] Dirac   --- Quantum Mechanics (4th Edition)
\item[] Bohm    --- Quantum Theory
\item[] Merzbacher --- Quantum Mechanics
\item[] Trigg   --- Quantum Mechanics
\item[] Gottfried --- Quantum Mechanics (Vol. I)
\item[] Kursunoglu --- Modern Quantum Theory
\item[] Landau and Lithschitz --- Quantum Mechanics 
\item[] Bethe and Jackiw --- Intermediate Quantum Mechanics
\item[] Jordan --- Linear Operators for Quantum Mechanics
\item[] Jauch --- Foundations of Quantum Mechanics
\item[] Pauling and Wilson --- Introduction to Quantum Mechanics
\item[] Powell and Crasemann -- Quantum Mechanics
\item[] Fano --- Mathematical Methods of Quantum Mechanics
\end{description}

There are also quite a few quantum mechanics books at the undergraduate
level.

\newpage

\underline{Index to 215A Notes}

\begin{description}
\item[I.] \underline{Introduction and Review}
\subitem $\Psi(\vec r, t)$ and $\Phi(\vec p, t)$ \hfill 3-I
\subitem $\Psi(\vec r, t) \rightleftharpoons \Phi(\vec p, t)$ via Fourier Transforms \hfill 5-I
\subitem The Hamiltonian and the Wave Equation \hfill 6-I
\subitem Probability Density and Probability Current \hfill 8-I
\subitem Scalar Products \hfill 9-I
\subitem Eigenvalue Equations \hfill 12-I
\subitem Schmidt Orthogonality Process \hfill 13-I

\item[II.] \underline{The Uncertainty Principle from Schwartz's Inequality}
\subitem Special case when $\Delta x \Delta p_x =\hbar/2$ \hfill 3-II

\item[III.] \underline{The Dirac Delta Function}
\subitem Common Representations of the $\delta$-function \hfill 2-III
\subitem 3-dimensional $\delta$-functions \hfill 4-III
\subitem Relations involving the $\delta$-function \hfill 7-III
\subitem Riemann-Stieltjes Integrals \hfill 9-III

\item[IV.] \underline{Wave Packets}
\subitem A Simple Wave Packet \hfill 2-IV
\subitem The Relation $\Delta x \Delta k \approx 1$ \hfill 4-IV
\subitem General Wave Packets \& Group Velocity
\subitem Spreading of Wave Packets

\item[V.] \underline{The Mathematical Framework of Quantum Mechanics}
\subitem Definition of a Field \hfill 3-V
\subitem Definition of a Linear Vector Space \hfill 4-V
\subitem Definition of a Unitary Space \hfill 6-V
\subitem Definition of a Hilbert Space \hfill 7-V
\subitem Linear Operators \hfill 9-V
\subitem Dirac's Bra and Ket notation \hfill 11-V
\subitem The Dual Space \hfill 13-V
\subitem Hermitian Operators \hfill 20-V
\subitem The Hermitian Conjugate of Expressions \hfill 22-V
\subitem Eigenvalues and Eigenspectra \hfill 23-V
\subitem Projection Operators \hfill 25-V
\subitem Closure Relations \hfill 27-V
\subitem Finite Matrices \hfill 29-V
\subitem Infinite Matrices \hfill 33-V
\subitem Matrix Representations of Quantum Mechanics \hfill 35-V
\subitem Unitary Transformations \hfill 40-V
\subitem Matrix Transformations \hfill 41-V
\subitem Change of Represenations \hfill 44-V

\item[VI.] \underline{The Physical Framework of Quantum Mechanics}
\subitem The Postulates of Quantum Mechanics \hfill 2-VI
\subitem The Schroedinger Picture \hfill 10-VI

\item[VII.] \underline{The One Dimensional Harmonic Oscillator}
\subitem The Harmonic Oscillator in the Schroedinger Picture \hfill 1-VII
\subitem Hermite Polynomials\hfill 5-VII
\subitem The Harmonic Oscillator to Bosons via Creation and Destruction Operators \hfill 10-VII
 
\item[VIII.] \underline{Bound States of Some Central Potentials}
\subitem The Orbital Angular Momentum Operators \hfill 2-VIII
\subitem The Spherical Harmonics, $Y_{LM}$\hfill 3-VIII
\subitem Bound States of a Square Well \hfill 7-VIII
\subitem Spherical Bessel Functions \hfill 7-VIII
\subitem Bound States of the Hydrogen Atom \hfill 10-VIII
\subitem The Isotropic Harmonic Oscillator \hfill 13-VIII

\item[IX.]\underline{Angular Momentum Concepts}
\subitem Raising and Lowering Operators, $L_+$ and $L_-$\hfill 2-IX
\subitem The Pauli Spin Matrices \hfill 7-IX
\subitem The Eigenkets of spin-$\half$ Particles \hfill 8-IX
\subitem Rotation of Scalar Fields \hfill 9-IX
\subitem The Rotation Operator, $e^{-iL_z \,d\theta}$\hfill 10-IX
\subitem The Generalized Rotation Operator, $e^{-iJ_z \,d\theta}$ \hfill 13-IX
\subitem Addition of Angular Momentum\hfill 17-IX
\subitem Clebsch-Gordan Coefficients \hfill 24-IX
\subitem The Singlet and Triplet States \hfill 27-IX
\subitem Symmetric and Antisymmetric States \hfill 28-IX
\subitem The Slater Determinant \hfill 31-IX
\subitem The $jj$ and $LS$ coupling Schemes\hfill 36-IX

\end{description}
\setcounter{section}{-1}

\input true_0.tex
\input true_1.tex
\input true_2.tex
\input true_3.tex

\section{Groups and Transformations}

{\bf Definition} The elements $x$ and $x^{-1}gx$, where $x$ and $g$ are
elements of a group $G$, are called conjugate elements in $G$: in other 
words, $x$ and $y$ are conjugate if and only if there exists an element 
$g\in G$ such that $y = g^{-1}xg$.

Theorem 6.4.1 The relation of conjugacy between elements of any group
$G$ is an equivalence relation. 
\begin{description}
\item[\it Reflexive.] Since $x = e^{-1}xe$, x is conjugate to itself.
\item[\it Symmetric.] If $y=g^{-1}xg$ we have $x=gyg^{-1} =
(g^{-1})^{-1}y g^{-1}$.
\item[\it Transitive.] If $y=g^{-1}xg$ and $z=h^{-1}yh$ we have
$z=h^{-1}g^{-1}xgh=(gh)^{-1}x(gh).$
\end{description}

It follows that the relation divides the elements of G into mutually
exclusive equivalence classes; these are called the {\it conjugacy classes}
in G. 

the following form
  \begin{eqnarray*}
   P_1 & = & (E_1,0,0,-E_1),      \\
   P_2 & = & (E_2,E_2\sin\theta\cos\phi,E_2\sin\theta\sin\phi,
               -E_2\cos\theta),   \\
   P_3 & = & (E_3,0,0,\beta E_3),      \\
   q   & = & (E_1 - E_2,-E_2\sin\theta\cos\phi,-E_2\sin\theta\sin\phi,
                    E_2\cos\theta - E_1).
  \end{eqnarray*}
%\begin{figure*}[hb]
%\epsfysize=6cm %%%% whatever vertical size you want in cm or inches
%\centerline{\epsfbox{user$1:[smith.tex.eps]feyn.eps}}
%\caption[fig1]{Definition of Kinematic Variables.}
%\label{fig:feyn}
%\end{figure*}
  \begin{eqnarray*}
    y\     & \approx & 1 - {E_2\over E_1} {(1+\cos\theta)\over2}, \\
    \qsq \ & \approx & {E_2\over E_1}{(1-\cos\theta)\over2}, \\
    x      & \approx & {Q^2\over sy}.
  \end{eqnarray*}
\[ \begin{array}{lll} 
  {\cal M}^{++}=({\cal M}^{22} + {\cal M}^{11})/2, 
& {\cal M}^{+0}=({\cal M}^{23} -i{\cal M}^{13})/\sqrt{2},
& {\cal M}^{+-}=({\cal M}^{22} - {\cal M}^{11})/2, \\ [8pt]
  {\cal M}^{0+}=({\cal M}^{32} +i{\cal M}^{31})/\sqrt{2}, 
& {\cal M}^{00}= {\cal M}^{33}, 
& {\cal M}^{0-}=({\cal M}^{32} -i{\cal M}^{31})/\sqrt{2}, \\ [8pt]
  {\cal M}^{-+}=({\cal M}^{22} - {\cal M}^{11})/2, 
& {\cal M}^{-0}=({\cal M}^{23} +i{\cal M}^{13})/\sqrt{2}, 
& {\cal M}^{--}=({\cal M}^{22} + {\cal M}^{11})/2. \\ [8pt]
\end{array} \] In the case where all the cross terms with $\epsilon^1$
vanish we have the particular case that is satisfied by our system of 
basis vectors
\[ \begin{array}{lll} 
  {\cal M}^{++}=({\cal M}^{22} + {\cal M}^{11})/2, 
& {\cal M}^{+0}= {\cal M}^{23}/\sqrt{2},
& {\cal M}^{+-}=({\cal M}^{22} - {\cal M}^{11})/2, \\ [8pt]
  {\cal M}^{0+}= {\cal M}^{32}/\sqrt{2}, 
& {\cal M}^{00}= {\cal M}^{33}, 
& {\cal M}^{0-}= {\cal M}^{32}/\sqrt{2}, \\ [8pt]
  {\cal M}^{-+}=({\cal M}^{22} - {\cal M}^{11})/2, 
& {\cal M}^{-0}= {\cal M}^{23}/\sqrt{2}, 
& {\cal M}^{--}=({\cal M}^{22} + {\cal M}^{11})/2. 
\end{array} \]

\input appendix1.tex
\input appendix2.tex

%\begin{thebibliography}{99}
%\bibitem{BCDMS} J.J.Aubert et al., EMC Collaboration,
% Nucl.Phys. B259(1985)189;\\
% A.C.Benvenuti et al., BCDMS Collaboration, Phys.Lett.B223(1989)485
%\bibitem{kaiserf}     F. Eisele, {\em First Results from the H1
%Experiment at HERA}, and
%                      F. Brasse, {\em The H1 Detector at HERA},
%                      Invited talks, Proceedings of the
%            26th International Conference on High Energy Physics,
%                   Dallas (1992) and DESY preprint 92-140 (1992)
%\bibitem{H1-249} I.\, Abt, J.R.\,Smith, {\em MC Upgrades to Study
%Untagged Events}, H1 Internal Note H1-249, (1992).
%H1-249 considered only the diagonal elements of the Photon Flux
%for photons coupled to Spin-1/2 particles.
%\end{thebibliography}
 %
 \end{document}
