\section{The Dirac Delta Function}
Consider the relation $\displaystyle y_i = \sum_j \, a_{ij} x_j$ rewritten as $\displaystyle y(i) = \sum_j a(i,j) x(j)$ which states that $y(i)$ is a linear combination of the $x(i)'s$.
If thse indices were continuous, we would have an expression of the form
$$f(x) = \int\, G(x,x') \, g(x') \, dx'.$$ One says that $f(x)$ is a linear functional of $g(x)$ and $G(x,x')$ is called the ``kernel'' which depends in general on both $x$ and $x'$. We see that it is linear
because if $f_1 = \int\, G g_1\,dx$ and $f_2 = \int\, G g_2\,dx$, then $f_1 + f_2 = \int\, G(g_1 + g_2)\, dx$. A good example of this is our Fourier transform (in one dimension)
$$\psi(x) = {1\over \sqrt{2\pi}}\int\, e^{ikx}i, \phi(k)\, dk.$$ Here $\psi(x)$ is a linear functional of $\phi(k)$ with a kernel of $\displaystyle {1\over\sqrt{2\pi}}e^{-ikx}$. \\
On the other hand, we could have considered $\displaystyle y(i) = \sum_j\, a(i,j) x(j)$ as a linear transformation. Similarly, 
$$f(x) = \int\, G(x,x') g(x')\,dx$$ could be considered as a linear transformation where $g(x)$ are vectors with a set of continuous indices and $G(x,x')$ being the matrix for the transformation.\\
In the discrete indices case, the ``identity'' transformation is given by $a(i,j) = \delta_{ij}$. But in the continuous indices case, there is no function of $x$ and $x'$ which does a similar thing. However, 
this doesn't bother physicists who define a function called the ``Dirac delta function'', $\delta(x-x')$, which replaces $G(x,x')$ such that $$f(x) = \int \, \delta(x-x') f(x')\, dx'.$$ Some of the more common forms 
of representation for the Dirac delta function are:
\begin{eqnarray*}
1.\ \delta(x) &=& \lim_{\epsilon\rightarrow 0} {\epsilon\over \pi (\epsilon^2 + x^2)}\\
2.\ \delta(x) &=& \lim_{\epsilon\rightarrow 0} {1\over {dg\over dx} } {1\over \epsilon(b-a)}
\end{eqnarray*}where $g(x)$ is any smooth monotonic function in $(a,b)$ with $g(a) = -\infty$ and $g(b) = +\infty$. 
\begin{eqnarray*}
3.\ \delta(x) &=& \lim_{N\rightarrow \infty} {\sin(Nx)\over \pi x}\\
4.\ \delta(x) &=& {1\over 2\pi} \int_{-\infty}^{+\infty} \, e^{\pm i\mu x}\, du\\
5.\ \delta(x) &=& \lim_{\epsilon\rightarrow 0} {e^{-x^2/\epsilon}\over \sqrt{\pi\epsilon}}\\
6.\ \delta(x) &=& \lim_{\epsilon\rightarrow 0} {\theta(x+\epsilon) - \theta(x-\epsilon)\over 2\epsilon} \rightarrow {d\theta(x)\over dx}\\
\end{eqnarray*}where 
\begin{eqnarray*}
\theta(x) &=& 0 \mbox{ for } x<0\\
\theta(x) &=& 1 \mbox{ for } x > 0.
\end{eqnarray*}
Let us look at the firs one, i.e., 
$\delta(x-x') = \lim_{\epsilon\rightarrow 0} D(x-x',\epsilon)$, where $\displaystyle D(x,\epsilon) = {\epsilon\over \pi (\epsilon^2 + x^2)}.$
$D(x,\epsilon)$ for three values of $\epsilon$ are shown in the Figure at the right. We see that as $\epsilon$ becomes smaller, $D$ becomes more peaked around $x=0$ and is 
``practically'' zero elsewhere. Now $\displaystyle \int_{-\infty}^{+\infty} \, D(x,\epsilon)\, dx = 1$ and is independent of $\epsilon$. \\
Now if $f(x)$ is continuous around $x\approx x'$, then for small $ \epsilon$ (sharply peaked around 0), we have
$$ \int_{-\infty}^{+\infty}\, f(x') D(x-x')\,dx' \approx f(x) \int\, D(x-x',\epsilon)\, dx' = f(x).$$
The above is not a proof. However, if $f(x)$ is bounded everywhere, one can show that $$lim_{\epsilon\rightarrow 0}\int_{-\infty}^{+\infty}\, f(x) D(x,\epsilon)\, dx = f(0).$$
This is just what we want the $\delta$ function to do and if it does it, the exact form of $(D(x,\epsilon)$ is not important.\\
We will assume that a $\delta$ function exists such that 
$$\int_{-\infty}^{+\infty}\, f(x') \delta(x-x')\, dx' = f(x).$$
Furthermore, we should observer that the $\delta$ function will only have meaning when it appears under the integral sign.\\
Later on in this section, we will make a brief digression into Riemann-Stieltjes integrals where we shall see that integrals like the above can be handled with 
more mathematical rigor. \\
Extension into 3 dimensions is ``straight forward''. 
$$\delta(\vec{r} - \vec{r'}) = \delta(x-x')\delta(y-y')\delta(z-z')$$ so that 
$$\int\, f(\vec{r'}) \delta(\vec{r} - \vec{r'}) d\vec{r'} = \int\, f(x',y',z') \delta(x-x')\delta(y-y')\delta(z-z')\, dx'dy'dz' = f(x,y,z) = f(\vec{r}).$$
From our work with Fourier transforms
\begin{eqnarray*}
\delta(\vec{r}-\vec{r'}) &=& {1\over (2\pi)^3}   \int_{-\infty}^{+\infty}\, e^{\pm i \vec{k}\cdot(\vec{r}-\vec{r'}) }\, d\vec{k}\\
\delta(\vec{p}-\vec{p'}) &=& {1\over (2\pi)^3} \int_{-\infty}^{+\infty}\, e^{\pm i \vec{r}\cdot(\vec{p}-\vec{p'}) }\, d\vec{r}.
\end{eqnarray*}
In spherical coordinates $(r.\theta, \phi)$,
$$\delta(\vec{r}-\vec{r'}) = {\delta(r - r')\over r^2} \sum_{l,m}\, Y_{lm}^*(\theta.\phi)Y_{lm}(\theta'.\phi'),$$ where the $Y_{lm}(\theta,\phi)$'s are the spherical harmonics.\\
Example: Let us use the 3rd representation above and show that 
\begin{eqnarray*}
\int_A^B\, f(x)\delta(x)\, dx &=& f(0) \mbox{ if } A \le 0 \le B\\
                                          &=& 0 \mbox{ otherwise.}
\end{eqnarray*} We have 
$$\int_A^B\, f(x) \lim_{N\rightarrow \infty} {\sin(Nx)\over \pi x}\, dx.$$ We not for $x$ not near zero and $N$ large, the $\sin(Nx)$ oscillates rapidly. Consequently we wouldn't expect a large contribution to the integral
from these regions as long as $f(x)$ is a ``smooth'' and ``reasonably well behaved'' function.\\
As $\displaystyle \lim_{x\rightarrow0} {\sin(Nx)\over x} = N$, we will have around $x=0$ just $\displaystyle \int_{x\approx0} \, N f(x)\, dx$. Thus our ``main'' contributions should come around
$x\approx0$ and it shouldn't be too surprising to get just $f(0)$.\\
Let us show this in detal. We let 
$$I = {1\over\pi}\lim_{N\rightarrow\infty} \int_A^B\, f(x) {\sin(Nx)\over x}\,dx.$$
Integrate by parts with $u=f(x)$ and $\displaystyle dv = {\sin(Nx)\over x}\,dx$.
$$I = {1\over\pi} \lim_{N\rightarrow\infty} \left\{\left[ f(x) \int_A^x\, {\sin(Ny)\over y} \, dy \right]_A^B  - \int_A^B f'(x)\,dx \int_A^x\, {\sin(Ny)\over y} \,dy \right\}.$$
WIth $z = Ny$, 
$$I = {1\over\pi}\lim_{N\rightarrow\infty} \left\{ \left[ f(x) \int_{NA}^{Nx}\, {\sin(Ny)\over y}\,dy\right]_A^B - \int_A^B f'(x)\,dx \int_{NA}^{Nx}\, {\sin(Ny)\over y}\,dy \right\}.$$
First, we look at the 1st term on the right hand side. For the lower limit, $Nx\rightarrow NA$ and the integral is zero. For the upper limit $Nx \rightarrow NB$. If $A$ and $B$ are both 
$\left\{\begin{matrix}
\mbox{positive} \\
\mbox{negative} 
\end{matrix}\right\}$, the integral is still zero as $\displaystyle \lim_{N=\infty} NA = \lim_{N \rightarrow \infty} NB = \{\pm \infty\}$.
If $A$ is negative and $B$ us positive, 
$$\lim_{N\rightarrow\infty}\int_{NA}^{NB}\, {\sin(z)\over z}\,dz = \int_{-\infty}^{+\infty} {\sin(z)\over z}\,dz = \pi.$$
So the 1st term $= f(B)$ if $A<0$ and $B>0$ and is zero otherwise.\\
Similar reasoning applies to the 2nd term. It will be zero unless $A$ is negative and $x$ is positive in the integral over $dy$. This means that in the integral over $dx$, we can ``neglect'' that
part it where $x<0$. So we replace $\int_A^B\,dx$ by $\int_0^B\, dx$. With this, the 2nd term becomes 
$$- {1\over\pi} \int_A^B f'(x)\,dx \left[ \lim_{N\rightarrow\infty} \int_{NA}^{Nx} \, {\sin(Ny)\over y} \,dy \right].$$
The bracket gives $\pi$ for all $x>0$ and the second term becomes $-f(B) + f(0)$. So 
\begin{eqnarray*}
I = \int_A^B\, f(x)\delta(x)\, dx &=& f(0) \mbox{ if } A<0 \mbox{ and } B>0\\
&=& 0 \mbox{ otherwise}
\end{eqnarray*}
End example.\\

\subsection{Relations involving the $\delta$-Function}
Some of the more ``useful'' relationships involving the $\delta$ function are: \hbox{(see Schiff pp 57)}
\begin{eqnarray*}
&1.~& \int\, \delta(x)\, dx  = 1\\
&2.~& \int\, \delta(-x)\, dx = \delta(x) \\
&\null&~~~~~~~~~i.e. \int_{-\infty}^{+\infty}\, f(x)\delta(x) \, dx = \int_{-\infty}^{+\infty}\, f(x)\delta(-x) \, dx = f(0)\\
&3.~& \delta(ax) = {1\over a} \delta(x) \mbox{ for } a> 0\\
&4.~& x\delta(x) = 0\\
&\null&~~~~~~~~~ i.e. \int_{-\infty}^{+\infty}\, f(x) x \delta(x) \, dx = f(x) x \big|_{x=0} = 0 \\
&5.~& \int\,\delta(x-x'')\delta(x''- x')\, dx'' = \delta(x-x')\\
&6.~& f(x)\delta(x-x') = f(x')\delta(x-x')\\
&7.~& \delta(x^2 - a^2) = {1\over2a} \left[\delta(x-a) + \delta(x+a)\right]\\
&8.~& x\delta'(x) = x{d\over dx}\delta(x) = -\delta(x)\\
&9.~& \delta^{(m)}(x) = (-1)^m \delta^{(m)}(-x)\\
&10.& \int\, \delta^{(m)}(x-y) \delta^{(n)}(y-a)\, dy = \delta^{(m+n)}(x-a)\\
&11.& x^{m+1}\delta^{(m)}(x) = 0\\
&12.& \int\, f(x) \delta^{(m)}(x)\, dx = (-1)^m f^{(m)}(0)\\
&\null&~~~~~~~~~~\mbox{ providing that } f^{(m)}(0) \mbox{ exists}.
\end{eqnarray*}
Now let us digress briefly and look at Riemann-Stieltjes Integrals.
We shall see that the Dirac $\delta$ function can be formulated in a ``rigorous'' manner in this case. 

\subsection{Digression on Riemann-Steltjes Integrals}
(cf. Chapter 9 of Mathematical Analysis by Apostol.)\\
\underline{Riemann Integrals}\\
Many physicists and scientist never see nor never use any integral besides the Riemann integral.\\
For a simplified picture of this integral, let us divide the interval $[a,b]$ up into $n$ parts, let $t_k$ be a point between $x_{k-1}$ and $x_k$, and let $\Delta x_k = x_k - x_{k-1}$. Then we normally think of 
the Riemann integral as 
$$\int_a^b\, f(x)\, dx = \lim_{n\rightarrow\infty} \sum_{k=1}^n\, f(t_k) \Delta x_k$$ where we have assumed that the limit exists.\\
Graphically, this integral represents the area under the curve $f(x)$. Note that $f(x)$ must be continuous but that it does not need to have a continuous first derivative for 
the integral to exist. \\
Let us now look at the Riemann-Stieltjes integral which will reduce to the Riemann integral in special cases. 
\subsection{Riemann-Steltjes Integrals}
First we make a few definitions and state some theorems without proof. I refer you to Apostol for details.\\
We will be considering two functions, $f(x)$ and $g(x)$ where both may or may not be continuous or even differentiable functions. 

\begin{definition} 
$P$ is called a partition of the interval $[a,b]$ and consists of the finite set of points \\  $a=x_0<x_1<x_2<\hdots<x_{n-1}<x_n = b$.
\end{definition}

\begin{definition}
The partition $P'$ of $[a,b]$ is ``finer'' than $P$ or a ``refinement'' of $P$ if $P$ is contained in $P'$. i.e., $P\subset P'$.
\end{definition}

\begin{definition}
Let $P= \{ x_o, x_1, x_2, \hdots x_n\}$ be a partition of $[a,b]$, $\Delta \alpha_k \equiv \alpha(x_k) - \alpha(x_{k-1})$, and $t_k$ be a point in $[x_{k-1}, x_k]$. The sum
$$S(P,f,\alpha) = \sum_{k=1}^n\, f(t_k) \Delta \alpha_k$$ is called a ``Riemann-Stieltjes Sum'' of $f$ with respect to $\alpha$ on $[a,b]$.
\end{definition}
Now we say that $f$ is Riemann integrable with respect to $\alpha$ on $[a,b]$, which we denote by `` $f\in R(\alpha)$ on $[a,b]$'', if there exists a number $A$ such that:\\
For every $\epsilon>0$, there exists a partition $P_\epsilon$ of $[a,b]$ such that for every partition $P$ finer than $P_\epsilon$ and for every $t_k$ in $[x_{k-1}, x_k]$ we have
$$\left| S(P,f,\alpha) - A\right| < \epsilon.$$
When this is the case, we denote $A$ by the integral $ \displaystyle \int_a^b\, f(x)\, d\alpha(x)$ or simply bu $\displaystyle \int_a^b \, f\, d\alpha$.\\
One can now show:\\
If $f\in R(\alpha)$ and $g\in R(\alpha)$, then for arbitrary constants $c_1$ and $c_2$, we have 
$$\int_a^b\, (c_1 f + c_2 g)\, d\alpha = c_1\int_a^b\, f\, d\alpha + c_2 \int_a^b\, g\, d\alpha.$$
Similarly, if $f\in R(\alpha)$ and $f\in R(\beta)$, then 
$$\int_a^b\, f\, d(c_1 f + c_2 g)  = c_1\int_a^b\, f\, d\alpha + c_2 \int_a^b\, f\, d\beta.$$ 
So the integral is ``linear'' in both $f$ and $\alpha$.\\
If $c \in [a,b]$ and $f\in R(\alpha)$, then 
$$\int_a^b\, f\, d\alpha = \int_a^c\, f\, d\alpha + \int_c^b\, f\, d\alpha.$$

\begin{definition}
If $a<b$, we define $\displaystyle \int_b^a \, f\, d\alpha = - \int_a^b\, f\, d\alpha$ whenever $\displaystyle \int_a^b\, f\, d\alpha$ exists. Also we define
$\displaystyle \int_a^b\, f\, d\alpha = 0$. 
\end{definition}

\begin{theorem}
If $f\in R(alpha)$ on $[a,b]$, then $\alpha \in R(f)$ on $[a,b]$ and 
$$\int_a^b \ f(x) \, d\alpha(x) + \int_a^b\, \alpha(x) \, df(x) = f(b)\alpha(b) - f(a)\alpha(a).$$
\end{theorem}

This ``formula'' or ``relationship'' is known as ``the formula for integration by parts'' and tells us that if $\displaystyle \int_a^b\, f\, d\alpha$ exists, then so does
$\displaystyle \int_a^b\, \alpha\, df$.\\
Note as we go along that everything we say also applies to the Riemann integral. However, these relationships are more powerful as we shall see shortly. In fact, 
$f$ and $\alpha$ need not be continuous and/or differentiable for the above to hold.

