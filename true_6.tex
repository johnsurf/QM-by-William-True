\section{\underline{Physical Framework of Quantum Mechanics}}

We have now discussed the more important mathematical aspects which we need to formulate quantum mechanics. Now we will give a set of postulates of quantum mechanics. This set of postulates are probably not complete. 
In fact, one may not be able to write down a set of postulates which covers everything that one might encounter in non-relativistic quantum mechanics. But these postulates will go ``a long ways'' towards setting up a framework
to work in. In fact, I do not know of a textbook which lists a set of postulates and continues from there. Messiah in Chapter VIII does the most complete job in this respect I strongly recommend that you read this chapter in his book. 
A word of caution however, when reading various books, one must be careful to distinguish between postulates and definitions. 

As you know, a classical system is described by a set of dynamical variables, e.q., position, momentum, energy, angular momentum. If these dynamical variables are known at any given instant
of time, then we can in principle determine them at any other instant of time with infinite precision. 

I repeat here the definition of an observable which we gave on \cpageref{observable}. 
\begin{definition} A Hermitian operator is called an observable if its eigenkets form a complete set and span the entire vector space.
\end{definition}

As I have pointed out before, it is usually not trivial to prove that a given set is complete. Consequently it is ``traditional'' for physicist to ignore this and to assume without proof that all Hermitian operators which
correspond to physical quantities are observables. Part of this is contained in Postulate 2 below. But I have never seen anyone prove that the corresponding set pf eigenkets is indeed complete.
The 1st postulate is: 
\begin{postulate}1. To every type of physical system, there corresponds an abstract Hilbert space and each vector in this space represents exactly one possible state of the system.
\end{postulate} 
This postulate tells us that the system is described by a vector in a complete unitary linear vector space and so the various states of the system linearly ``superposable''. In addition, this ``state vector'' describing the system 
need not be constant but can move about in our Hilbert space as the system evolves in time -- See Postulate 5 below. 

In classical physics, the dynamical state of a system is described by a point in phase space and conversely each point in phase space represents a different state of the system. The same is also true for our Hilbert space with 
the exception that two vectors which are multiples of each other represent the same state. 

The second postulate is: 
\begin{postulate} 2. To each dynamical variable we associate a Hermitian linear operator which is an observable.
\end{postulate}
Now we have seen that not all operators in quantum mechanics will commute with each other. Those operators which do commute are said ``to be compatible''. If for a given system, we found all the compatible
dynamical variables, we would say that we had \underline{a complete set of compatible variables}. The precise determination of these variables of this complete set will give us the maximum information that we can obtain for 
this given quantum mechanical system. -- Not that this is less information than one can obtain for a classical system since there one can determine all variables precisely and not just the compatible ones. 

The 3rd postulate is: 
\begin{postulate}3. To any dynamical variable of a system, one can always add a certain number of other dynamical variables and thus form a complete set of compatible variables.
\end{postulate}
In choosing this complete set of compatible variable, one is usually ``guided'' by the dynamical variables of the corresponding classical system. Does one know when he has them all? Generally not for a reasonably complicated 
system. Sometimes a variable in a quantum mechanical system does not have a classical analogue. -- For example the intrinsic spin of a particle.