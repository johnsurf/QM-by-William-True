\section{\underline{Physical Framework of Quantum Mechanics}}

We have now discussed the more important mathematical aspects which we need to formulate quantum mechanics. Now we will give a set of postulates of quantum mechanics. This set of postulates are probably not complete. 
In fact, one may not be able to write down a set of postulates which covers everything that one might encounter in non-relativistic quantum mechanics. But these postulates will go ``a long ways'' towards setting up a framework
to work in. In fact, I do not know of a textbook which lists a set of postulates and continues from there. Messiah in Chapter VIII does the most complete job in this respect I strongly recommend that you read this chapter in his book. 
A word of caution however, when reading various books, one must be careful to distinguish between postulates and definitions. 

As you know, a classical system is described by a set of dynamical variables, e.q., position, momentum, energy, angular momentum. If these dynamical variables are known at any given instant
of time, then we can in principle determine them at any other instant of time with infinite precision. 

I repeat here the definition of an observable which we gave on \cpageref{observable}. 
\begin{definition} A Hermitian operator is called an observable if its eigenkets form a complete set and span the entire vector space.
\end{definition}

As I have pointed out before, it is usually not trivial to prove that a given set is complete. Consequently it is ``traditional'' for physicist to ignore this and to assume without proof that all Hermitian operators which
correspond to physical quantities are observables. Part of this is contained in Postulate 2 below. But I have never seen anyone prove that the corresponding set pf eigenkets is indeed complete.
The 1st postulate is: 
\begin{postulate}1. To every type of physical system, there corresponds an abstract Hilbert space and each vector in this space represents exactly one possible state of the system.
\end{postulate} 
This postulate tells us that the system is described by a vector in a complete unitary linear vector space and so the various states of the system linearly ``superposable''. In addition, this ``state vector'' describing the system 
need not be constant but can move about in our Hilbert space as the system evolves in time -- See Postulate 5 below. 

In classical physics, the dynamical state of a system is described by a point in phase space and conversely each point in phase space represents a different state of the system. The same is also true for our Hilbert space with 
the exception that two vectors which are multiples of each other represent the same state. 

The second postulate is: 
\begin{postulate} 2. To each dynamical variable we associate a Hermitian linear operator which is an observable.
\end{postulate}
Now we have seen that not all operators in quantum mechanics will commute with each other. Those operators which do commute are said ``to be compatible''. If for a given system, we found all the compatible
dynamical variables, we would say that we had \underline{a complete set of compatible variables}. The precise determination of these variables of this complete set will give us the maximum information that we can obtain for 
this given quantum mechanical system. -- Not that this is less information than one can obtain for a classical system since there one can determine all variables precisely and not just the compatible ones. 

The 3rd postulate is: 
\begin{postulate}3. To any dynamical variable of a system, one can always add a certain number of other dynamical variables and thus form a complete set of compatible variables.
\end{postulate}
In choosing this complete set of compatible variable, one is usually ``guided'' by the dynamical variables of the corresponding classical system. Does one know when he has them all? Generally not for a reasonably complicated 
system. Sometimes a variable in a quantum mechanical system does not have a classical analogue. -- For example the intrinsic spin of a particle.

Disagreement with experimental results is one way one can tell that our set is incomplete. Just how one discovers and/or decides on  what the ``missing'' variables are is basically a matter of intuition and/or insight and/or
educated guesswork, etc. 

Remember that the uncertainty principle tells us that non-compatible variables cannot be precisely determined at the same time: 

THe 4th postulate is:
\begin{postulate}4. If the system is in a state described by $\ket{u}$, the mean value of a physical quantity corresponding to the operator $A$ is given by
$$\expect{A} = {\xAy{u}{A}{u}\over \braket{u}{u} }.$$
\end{postulate}
This postulate tells us that we cannot tell the difference between two state vectors which are multiples of each other. In addition, it tells us that all states are undetermined up to an arbitray phase. 

The 5th and final postulate is: 
\begin{postulate}5. Let $\ket{\Psi(t)}$ describe the state of the system at time $t$ and let $H$ be the Hamiltonian operator of the system. The system will evolve in time according to the equation
$$H\ket{Psi(t)} = i\hbar{d\over dt} \ket{Psi(t)}.$$
\end{postulate}
Because this equation above is of first order in time, the linear superposition of states is preserved in the course of time. Messiah shows in Chapter 8, paragraph 8 that this equation follows quite naturally from the 
conservation of linear superposition of states if the system is conservative -- i.e. $H$ is not an explicity function of the time. However, we postulate that this equation is still valid even though $H$ may be time dependent. 

Let us now make some additional remarks concerning this complete ste of compatible observables which we need. 

1. If the system has a classical analogue, we can take for the ``fundamental observables''
the $3N$ coordinates $q_i$ and the $3N$ corresponding conjugate momenta, $p_i$, along with the commutation relations:
$$[q_i,q_j] = 0, \quad [p_i,p_j] = 0, \mbox{ and } [q_i,p_j] = i\hbar\delta_{ij}.$$
All other observables, even $H$, can be written in terms of the $p$'s and $q$'s providing one properly symmetrizes ``ambiguos'' operators 
$$\mbox{e.g. } p_i q_i \Rightarrow {p_iq_i + q_ip_i\over 2}.$$

2. If the system does not have a classical analogue, one usually has to introduce additional variables to those above and to specify the commutation relations among themselves and with the other observables which have a 
classical analogue. As I have pointed out above, finding these additional variables is not always easy and one does not know if he has them all. For example, the ground state of the hydrogen atom may be degenerate and 
we don't know it. 

Let us suppose that we have somehow found this complete set of compatible variables, $A, B, C, \hdots$. Then in principle, we can find eigenstates $\ket{a,b,c,\hdots}$ which are simultaneous 
eigenkets of these compatible variables and which form a basis for our Hilbert Space. 

Now our Hilbert Spaces can vary greatly depending on the system we are describing. As an example, consider a system containing a single particle with spin of $\hbar/2$ which can exist in only one of two states, one state
with ``spin up'' and the other state with ``spin down''. In this case, our Hilbert Spaces is a 2-dimensional space with basis vectors $\ket{+\hbar/2}$ and $\ket{-\hbar/2}$.

On the other hand, consider a particle moving in a 1-dimensional spaces. Here the basis could be made of up the eigenkets, $\ket{q}$, of the position operator (or of the eigenkets, $\ket{p}$, of the momentum operator). For this 
simple system, out Hilbert Space is an infinite-dimensional space. 

We have seen that when our eigenkets can be labeled by discrete indices 
$$\braket{n_i}{n_j} = \delta_{ij}$$ and that when they are labeled y continuous indices, 
$$\braket{q}{q'} = \delta(q-q').$$
Now it may seem odd to have a scalar product come out as a delta-function.

When we make the transition from our abstract Hilbert space to the coordinate representation (the Schr\"odinger Picture), we shall see that these $\delta$-functions will always appear under the integral signs and thus
will ``give us no trouble''.

Before we can make this transition to the coordinate representation, let us show that 
$$\xAy{q'}{p}{q''} = {\hbar\over i} \delta'(q-q'') \mbox{ where } \delta'(x) = {d\over dx}\delta(x).$$
In the coordinate representation, $\ket{q'}$ is an eigenket of the operator $q$ with an eigenvalue of $q'$, i.e., $q\ket{q'} = q'\ket{q'}$. 

We now consider the Unitary Operator $\ds S(\epsilon) = \exp^{-{iq\epsilon\over\hbar}}$. Using $[q,p] = i\hbar$, it is quite easy to show that 
$$[q,S(\epsilon)] = \epsilon S(\epsilon).$$ 
So $qS(\epsilon) = S(\epsilon) q + \epsilon S(\epsilon) = S(\epsilon)(q + \epsilon)$. Then we see that 
$$q[S(\epsilon)\ket{q'}] = S(\epsilon)(q+\epsilon)\ket{q'} = S(\epsilon)(q' + \epsilon)\ket{q'} = (q' + \epsilon)[S(\epsilon\ket{q'}].$$
So $S(\epsilon)\ket{q'} = \ket{q'+ \epsilon}$ and this $\ds \xAy{q'}{S)\epsilon}{q''} = \braket{q'}{q''+\epsilon} = \delta(q'-q''-\epsilon)$. 
When $\epsilon$ is an infinitetesimal quantity, $\ds S(\epsilon) \approx 1 - {i p\epsilon\over \hbar}$ and then 
$$ \xAy{q'}{S(\epsilon)}{q''} \approx \xAy{q'}{1 - {i p\epsilon\over \hbar}}{q''} = \braket{q'}{q''} =  {i \epsilon\over \hbar}\xAy{q'}{p}{q''}$$ 
or that 
$$\xAy{q'}{p}{q''} \approx {\hbar\over i} \left[ {\delta(q'-q'') - \delta(q'-q''-\epsilon)\over \epsilon}\right]$$ and in the limit as $\epsilon\goes 0$, we have 
$$\xAy{q'}{p}{q''} = \lim_{\epsilon\goes 0} {\hbar\over i} \left[ {\delta(q'-q'') - \delta(q'-q''-\epsilon)\over \epsilon}\right] = {\hbar\over i} \delta'(q'-q'').$$ \label{10-VI}
In a similar manner, one can show
$$\xAy{p'}{q}{p''} = i\hbar \delta'(p'-p'').$$
 
\subsection{\underline{The Schr\"odinger Picture} or \\ \underline{Wave Mechanics in the Coordinate Representation}}

What we will now show is that the Schr\"odinger Picture is nothing more than our quantum mechanics in our abstract Hilbert space in the coordinate representation. 
To do this, will consider a system which has a classical analogue with the 3 fundamental observables being the $3N q_i$'s and the $3N p_i$'s where 
$$[q_i, q_j] = 0, \quad [p_i, p_j] = 0, \mbox{     and } [q_i, p_j] = i\hbar \delta_{ij}.$$
The coordinate representation consists of picking as a basis the eigenkets of the $3N q_i$ operators, i.e,
$$q_i'\ket{q'_1. q'_2, q'_3, \hdots} = q'_i \ket{q'_1. q'_2, q'_3, \hdots}.$$

Let us use the following notation:
\begin{eqnarray*}
\ket{q'} &=& \ket{q'_1, q'_2, q'_3, \hdots}\\
dq'      &=& dq'_1\,dq'_2\,dq'_3\,\hdots\\
\delta(q'-q'') &=& \delta(q'_1 = q''_1) \delta(q'_2 = q''_2)\delta(q'_3 = q''_3)
\end{eqnarray*}
Then our closure relation is given by
$$P = \int\, \ket{q'}\, dq'\, \bra{q'} = I.$$
If $\ket{\psi}$ is a state vector in our Hilbert space describing a given state of the system, we can write using $P$
$$\ket{\psi} = \int\, \ket{q'} \, dq' \braket{q'}{\psi}.$$
We have seen in the matrix representation that $\braket{q'}{\psi}$ are elements of a column matrix for $\ket{\psi}$. In this case, the elements of this column matrix 
vary with a continuous index $q'$. Since $\braket{q'}{\psi}$ is a scalar which varies continuously with $q'$ and ``anticipating what is to come'', we define $\psi(q') \equiv \braket{q'}{\psi}$ -- 
i.e., $\psi(q')$ is a function of $q'$. Remember that our condensed notation real means
$$\braket{q'}{\psi} = \bra{q'_1 q'_2 q'_3\hdots}{\psi} = \psi(q') = \psi(q'_1, q'_2, q'_3, \hdots).$$

Now let us look at the scalar product $\braket{\psi_1}{\psi_2}$.
\begin{eqnarray*}
\braket{\psi_1}{\psi_2} &=& \left[ \int\, \braket{\psi_1}{q'} \bra{q'}\, dq'  \right]  \left[  \int\,  \ket{q''}\braket{q''}{\psi_2} \, dq''   \right] \\
                                    &=& \int\, \int\, \braket{\psi_1}{q'} \braket{q'}{q''} \braket{q''}{\psi_2} \,dq'\, dq'' \\
                                    &=& \int\, \int\, \braket{\psi_1}{q'} \delta(q' -q'') \braket{q''}{\psi_2} \,dq'\, dq''\\
                                    &=& \int\, \braket{\psi_1}{q'}\braket{q'}{\psi_1} \,dq' = \int\, \psi^*_1(q') \psi^*_2(q') \,dq'
\end{eqnarray*}

Thus we see that in the coordinate representation, the scalar product $\braket{\psi_1}{\psi_2}$  is the same scalar product that we would use in the Schr\"odinger Picture. 

Next we look at some operator relationships involving the $q_i$'s. The simplest is with $q_i$ operating on $\ket{\psi}$. Consider $\ket{\chi} \equiv q_i \ket{\psi}$. 
$$\ket{\chi} = q_i\ket{psi} = \int\, \ket{q'}\, dq'\, \xAy{q'}{q_i}{\psi} = \int\, \ket{q'}\, dq'\, q' \braket{q'}{\psi} $$ where we let the operator $q_i$ operator to the left which we can 
do as it is a Hermitian operator. The components of $\ket{\chi}$ are
\begin{eqnarray*}
\chi(q'') = \braket{q''}{\chi} &=&\int\, \braket{q''}{q'}\, dq'\, \braket{q'}{\psi} = \int\, \delta(q''-q')\, dq'\, q'_i\braket{q'}{\psi} \\
&=& q''_i \psi(q'').
\end{eqnarray*}
Similarly, for any Hermitian operator which is an arbitrary function of the $q_i$'s, we have
for $V(q)$ that $\xAy{q'}{V(q)}{\psi} = V(q')\braket{q'}{\psi} = V(q')\psi(q')$ and this is just what we have or ``use'' in the Schr\"odinger Picture. 

Now let us see what happens when one of the $p_i$'s is our operator. Let $\ket{\chi} \equiv p_i\ket{\psi}$. THen 
$$\braket{q'}{\chi} = \xAy{q'}{p_i}{\psi} = \int\, \xAy{q'}{p_i}{q''}\braket{q''}{\psi}\, dq''.$$
We saw on page \cpageref{10-VI} that 
$$\xAy{q'_i}{p_i}{q_i''} = {\hbar\over i} \delta'(q_i'-q_I'')  = \lim_{\epsilon\goes 0} {\hbar\over i} \left[ {\delta(q'-q'') - \delta(q'-q''-\epsilon)\over \epsilon}\right].$$ 
Because of our condensed notation, we have here 
$$\xAy{q'}{p_i}{q''} = {\hbar\over i} \delta'(q_1'-q_1'')\delta'(q_2'-q_2'')\hdots\delta'(q_{i-1}'-q_{i-1}'')\delta'(q_i'-q_i'') \delta'(q_{i+1}'-q_{i+1}'') \hdots.$$ 
and we can easily integrate over all the $dq'' = dq''_1dq''_2\hdots$ except the $q''_i$ one. With $\braket{q''}{\psi} \equiv \psi(q'')$, we have
\begin{eqnarray*}
\braket{q'}{\chi} &=& \lim_{\epsilon\goes 0} {\hbar\over i} \left[{\int\, \psi(q'')\delta(q'_i-q''_i)\, dq''_i - \psi(q'')\delta(q'_i-q'_i -\epsilon)\,dq''_i\over \epsilon}\right]\\
                         &=& \lim_{\epsilon\goes 0} {\hbar\over i} \left[{\int\, \psi(q'_1,q'_2,q'_3,\hdots,q'_i\hdots) - \psi(q'_1,q'_2,q'_3,\hdots,q'_i-\epsilon\hdots) \over \epsilon}\right] ={\hbar\over i} {\partial\over\partial q'_i}\psi(q')
\end{eqnarray*}
So we see that 
$$\braket{q'}{\chi} = \xAy{q'}{p_i}{\psi} = {\hbar\over i} {\partial\over\partial q'_i}\psi(q').$$
and this is just how the operator $p_i$ is ``supposed to behave'' in the Schr\"odinger Picture. 

Clearly we can generalize these results so that for any operator $A(p,q)$ operating on $\ket{\psi}$ will be represented in the Schr\"odinger Picture by $\ds A\left( {\hbar\over i} {\partial\over\partial q}, q\right)$ operating 
on $\psi(q)$. 

Furthermore, the equation 
$$H\ket{\psi(t)} = i\hbar{d\over dt}\ket{\psi(t)}$$ clearly can be ``replaced'' by 
$$H \psi(t) = i\hbar{\partial \over \partial t} \psi(t)$$ where $\ds {d\over dt} \goes {\partial \over \partial t}$ because the eigenvalues of the operator $q$ appear explicitly in the component $\psi(q,t)$. 

So we are now back to the Schr\"odinger Picture. But we should stress that this is just quantum mechanics in a given representation -- the coordinate representation. This is just one of the many valid
representations which one could work in. Another common representation is the momentum representation. Here the operator $p_i$ is replaced by its eigenvalue $p_i$ and the 
operator $q_i$ is replaced by $\ds -{\hbar\over i}{\partial \over \partial p_i}$ where it operators on $\braket{p'}{\psi} \equiv \phi(p')$. 

In principle, one need not even go into a representation to find out all there is to know about a quantum mechanical system as I will show shortly for the simple case of the linear harmonic oscillator.

One might say that we are now in the same position as a student studying E \& M who has been given Maxwell's Equations -- all there is left to do is to apply them to various systems with 
various boundary conditions. So for the rest of the year expect for relativistic quantum mechancs and the Dirac equation, we will look at various systems and learn techniques on how to ``solve'' them and
when to use different approximation. 

Usually we will work in the Schr\"odiinger Picture but we need not to do so. 







