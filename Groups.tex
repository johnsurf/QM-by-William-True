\section{Groups and Transformations}

{\bf Definition} The elements $x$ and $x^{-1}gx$, where $x$ and $g$ are
elements of a group $G$, are called conjugate elements in $G$: in other 
words, $x$ and $y$ are conjugate if and only if there exists an element 
$g\in G$ such that $y = g^{-1}xg$.

Theorem 6.4.1 The relation of conjugacy between elements of any group
$G$ is an equivalence relation. 
%\begin{description}
\begin{itemize}
\item[\it Reflexive.] Since $x = e^{-1}xe$, x is conjugate to itself.
\item[\it Symmetric.] If $y=g^{-1}xg$ we have $x=gyg^{-1} =
(g^{-1})^{-1}y g^{-1}$.
\item[\it Transitive.] If $y=g^{-1}xg$ and $z=h^{-1}yh$ we have
$z=h^{-1}g^{-1}xgh=(gh)^{-1}x(gh).$
\end{itemize}

It follows that the relation divides the elements of G into mutually
exclusive equivalence classes; these are called the {\it conjugacy classes}
in G. 