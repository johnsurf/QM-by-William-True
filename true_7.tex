\section{\underline{One Dimensional Harmonic Oscillator}}

1. \underline{Review}

First let us review the standard treatment of the linear harmonic oscillator in the coordinate representation which you can find in practically all text books of Quantum Mechanics, e.g, Pauling \& Wislon or Schiff pp. 67 ff. I will assume that you 
have already done this in a more elementary course in Quantum Mechanics and so I'll only review the more important aspects here. 

The harmonic oscillator or something closely related to it is forever appearing the physics of all types of phenomena -- mechanics, E \& M, solid state, etc. -- and is often used as the starting point to solving more complicated sytems. 

Our Hamiltonian is
$$H = {p^2\over 2m} + \half k x^2 = {p^2\over 2m}  + \half  m \omega^2 x^2  \mbox{   where  } \omega^2\equiv {k \over m}.$$
Since $H$H is time independent, we can write $\ds \Psi(x,t) = \psi(x)e^{-{iEt\over\hbar}}$ and our wave equation 
$$H\Psi(x,t) = i\hbar{\partial \over \partial t}\Psi(x,t)$$ reduces to 
$$H\psi(x) = E \psi(x) \mbox{   or  } \left( {\hbar^2\over 2m}{d^2\over dx^2} + E - \half m \omega^2 x^2 \right)\psi(x) = 0.$$
At this point, it is convenient to change variables and we define
$$E = \lambda\left({\hbar\omega\over 2}\right) \mbox{    and    } x = \xi\sqrt{ {\hbar\over m \omega}}.$$
Our differential equations becomes
$${d^2 \psi(\xi)\over d\xi^2} + (\lambda - \xi^2)\psi(\xi) = 0.$$
For large $\xi$ we can neglect $\lambda$ and we see that our solution goes like $\ds e^{-{\xi^2\over 2}}$. So we look for a solution of the form
$$\psi(\xi) = \phi(\xi)\exp^{-{\xi^2\over 2}}$$ which gives us a differential equation of 
$${d^2\phi\over d\xi^2} - 2\xi {d\phi\over d\xi} + (\lambda - 1)\phi = 0.$$
Now a standard approach is to try a series solution of the form
$$\psi(\xi) = \xi^s(b_o + b_1 \xi + b_2\xi^2 \hdots)\mbox{   where } b_0 \ne 0.$$
Substituting this solution into the differential equation we find by equating each coefficient of each power of $\xi$ to zero that $s = 0\mbox{  or } 1$ from the 1st equation and so $s=0 \mbox{ and/or } b_1 =0$ from the 2nd
equation. The succeeding equations five use recursion relations so that we have
\begin{eqnarray*}
\phi(\xi) &=& a_0 + a_2 \xi^2 + a_4 \xi^4 + \hdots \mbox{   for } s = 0  \mbox{ and}\\ 
\phi(\xi) &=& a_1\xi + a_3 \xi^3 + a_5 \xi^5 + \hdots \mbox{   for } s = 1  \mbox{ with}\\
a_{k+2} &=& {2k -\lambda + 1\over (k+1)(k+2)} a_k.
\end{eqnarray*}

One can quite easily show that both solutions $\ds \psi(\xi) = \exp^{-{lxi^2\over2}}\phi(xi)$ $\mbox{ for } s=0 \mbox{ or } 1$ diverge as $x\goes \pm \infty$.But this is not acceptable as $\psi(x)$ must remain finite.
$\psi(x)$ can only remain finite if $\phi(\xi)$ is a finite series which will occur if $\lambda$ is restricted to only take on the values $2n + 1$ where $n=0,1,2,\hdots$. This restricts the possible values of $E$ and we find
the quantized energy eigenvalues of $$E_n = (n + \half)\hbar\omega.$$
By picking the two arbitrary constants $a_0$ and $a_1$ ``correctly'', the $phi(\xi)$'s become the Hermite polynomials:
\begin{eqnarray*}
H_0(\xi) &=& 1\\
H_1(\xi) &=& 2\xi\\
H_2(\xi) &=& 4\xi^2 - 2i\\
H_3(\xi) &=& 8\xi^3 - 12\xi2i\\
\hdots &\null& \hdots\\
H_n(\xi) &=& (-1)^n \exp^{{\xi^2\over 2}}{d^n e^{-\xi^2}\over d\xi^n}.
\end{eqnarray*}
$H_n(\xi)$ satisfies the differential equation of $\phi(\xi)$ with $\lambda = 2n+1$, viz. 
$$H_n'' - 2\xi H'_n + 2n H_n = 0.$$
One can also show that 
\begin{eqnarray*}
H'_n &=& 2n H_{n-1}\\
0 &=& H_{n+1} - 2\xi H_n + 2n H_{n-1}\\
H_{n+1} &=& 2\xi H_n - H'_n
\end{eqnarray*}