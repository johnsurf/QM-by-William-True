\section{Wave Packets}
(cf. Schiff, Messiah, Merzbacher, Gottfied)\\
In quantum Mechanics, we describe particles by waves of by combinations of waves called wave packets. These wave packets should become more and more localized as one approaches the ``classical mechanical region of validity.''\\
In elementary treatment of scattering, one uses plane waves to describe a steady stream of particles incident on a ``target'' and then scattering. In more formal treatment of scattering, uses of plane waves result in the appearance of 
mathematical singularities. But if one uses wave packets, as they should, these difficulties disappear. We will see this later in on the year when we consider formal scattering theory.\\
Using wave packets to describe particles, the inherent ``wave properties'' will not allow us to make ``too precise'' statements about position, momentum, etc. and one is forced to do some ``hand waving.''
\subsection{A Simple Wave Packet}
In order to gain some insight into all this, let us study some simple wave packets in one dimension in an isotropic non-absorptive medium. Let us consider a disturbance or a pulse which is localized and moving. 
We would like to have this ``packet'' describe a moving particle. \\
The plane wave, $e^{i(kx - \omega t)}$, will have a phase velocity $v_\phi = \omega/k$, where in general $\omega$ is a function of $k$;. Now we could always analyze thie above wave packet in terms of these plane waves -- which we
attest to ``understand'' and ``know how'' to work with them. And it seems reasonably clear that the more ``k-components'' needed in a Fourier analysis approach, the more localized the wave packet will be and conversely.\\
We can gain some insight into all this by considering a packet built ``equally'' from plane waves with $k$ between $k-\Delta k$ and $k + \Delta k$ where $\Delta k \ll k$. In this case, out wave packet is 
$$\Psi(x,t) = \int_{k-\Delta k}^{k + \Delta k}\, e^{i(k'(x-x_0) - \omega't)}\, dk'$$ where $\omega'$ means that $\omega' \defeq \omega(k')$.\\
Suppressing momentarily the $\omega't$ part, we look at 
\bearray
\psi(x) &=& {e^{ik(x-x_0)}\over i(x-x_0)} \left[ e^{i\Delta k(x-x_0)} - e^{-i\Delta k(x-x_0)} \right]\\
&=& 2{\sin[\Delta k(x-x_0)]\over (x-x_0)} e^{ik(x-x_0)}
\eearray
The real part of $\psi(x)$ is shown at the right and looks something like a wave packet -- if it was moving.\\
A more general approach would be to multiply out exponential by some sort of weighting function, $f(k')$, which is large near $k'=k$ and small elsewhere and then integrate of $k'$. 
e.g. Figures go here -- where we did the 2nd example above. \\
Now $$|\psi(x)|^2 = {4 \sin^2[ \Delta k(x-x_0)]\over (x-x_0)^2}$$  and we see that $\psi|^2$ is small when $\displaystyle (x-x_0)\gg {1\over \Delta k}$ where $\Delta k$ is a ``measure'' of the spread of the wave packet
in $k$-space (momentum space).\\

\subsection{The Relation $\Delta x \Delta k \approx 1$}
We required (cf pp 2-4) that $\Delta k \ll k$. So whenever $(x-x_0) \gg {1\over \Delta k}$, then $(x-x_0) \gg {1\over k} = {\lambda\over2\pi}$ which tells us that $x$ is many wavelengths ``out from'' $x_0$. Or 
stating it all inversely, $\psi|^2$ will be large only when $$\Delta x\Delta k \approx 1$$ where $\Delta x = x-x_0$.\\
Now this is not an exact statment and we really can't make one when we are dealing with wave packets. All we can say is that $\psi|^2$ will be large when $\Delta x\Delta k\approx 1$. This implies that 
if $\Delta k$, the spread in momentum space, is small, then the spread in coordinate space, $\Delta x$, will be large and conversely. This result is also consistent with our uncertainty principle which was
$\Delta x\Delta p \ge \hbar/2$ or $\Delta x\Delta k \ge 1/2$.

\subsection{General Wave Packets \& Group Velocity}
Let us now concoct a traveling wave packet by using the weighting function $\displaystyle f(k') = A(k') e^{i\alpha'}$ where $A(k')$ is real and $\alpha' \defeq \alpha(k')$. We have
$$\Psi(x,t) = \int_{-\infty}^{\infty}\, e^{i(k'x - \omega't)}\, dk' = \int_{-\infty}^\infty\, A(k') e^{i(k'x - \omega't + \alpha')}\, dk'$$ where $A(k')$ is expected to be large only around $k\approx k'$. Note in passing that this is just
the Fourier transform of the packet, $\Phi(k',t) = \sqrt{2\pi}\, f(k') e^e{-i\omega't}$ in the momentum representation.

We rewrite $\Psi(x,t)$ as 
$$\Psi(x,t) = \int\, A(k') e^{i\phi'}\, dk'$$ where $\phi' \defeq k'x -\omega't + \alpha'$. If $\phi' = \phi(k')$ changes rapidly with $k'$, $|\Psi(x,t)|$ will not be large even though $A(k')$ may be large, i.e., $e^{i\phi'}$ oscillates so
rapidly, there will be a great deal of cancelation and very little contribution to the integral. Conversely, $|\Psi(x,t)|$ will be large -- and this is where the wave packet will be in the region where $\phi'$ is stationalry with respect 
to $k'$. That is, the ``center'' of our wave packet will be where
$${d\phi(k')\over dk'}\big|_{k'=k} = {d\phi'\over dk'}\big|_{k'=l} = 0.$$

Let us use the notation, $\displaystyle {d\phi'\over dk'}\big|_{k' = k} \defeq {d\phi\over dk}, {d\alpha'\over dk'}\big|_{k'=k} = {d\alpha\over dk},$ etc. Then we have
$${d\phi\over dk} = x -{d\omega\over dk} t + {d\alpha\over dk} = 0.$$ Since $\displaystyle {d\alpha\over dk}$ is a constant, we see that our wave packet moves to the right with a velocity 
$\displaystyle v_g \defeq {d\omega\over dk}$ which we call the ``group velocity''. 

\subsection{A Wave Packet Striking a Potential Step}
\underline{Example:} As a simple example, let us consider what happens when a wave packet is incident from the right on the potential step shown in the Figure.

Out time independent wave equation is
$$H\psi(x) = \left[{p^2\over 2m} + V(x)\right] \psi(x) = E\psi(x).$$
Let $U(x) \defeq {2mV(x)\over \hbar^2}, \E \defeq {2mE\over \hbar^2}$, and using $\displaystyle p^2 = -\hbar^2 {d^2\over dx^2}$,
our wave equation reduces to 
$${d^2\psi\over dx^2} + (\E - U)\psi = 0.$$
$\E$ can be greater than $U_2$ or in between $U_2$ and $U_1$. We treat these two cases separately.\\
\underline{Case 1: $U_1 < U_2 < \E$} The plane wave solutions in the two regions can be written as (cf. for example, Messiah pp 81)
$$\left.\begin{matrix}
\psi_1 = e^{-ik_1x} + R e^{ik_1x} \\
\psi_2 = S e^{-ik_2x} 
\end{matrix}\right\}
\mbox{ where }
\left\{\begin{matrix}
k_i = \sqrt{\E - U_i} \\
R = {k_1-k_2\over k_1 + k_2}\\
S = 1+R = {2k_1\over k_1 + k_2} 
\end{matrix}\right. $$
These solutions are the ones which would be found for a steady stream of particles incident on the potential step from the right.

In order to study how a wave packet will behave, we will take a linear combination of the $\psi$'s centered around the energy $\E$ -- really around $k$, which is the more relevant parameter in this case. 
We can construct wave packets by considering a real function $f(k_1' - k_1)$ which is peaked around $k_1' = k_1$. Then in region 1 we can construct the wave packet
\bearray 
\Psi_1(x,t) &=& \int_0^\infty\, f(k_1'-k_1) \psi_1 e^{-i{E't\over \hbar}}\, dk_1'\\
&=& \int_0^\infty\, f(k_1' - k_1) e^{-(k_1'x + {E't\over \hbar})}\, dk_1' + \int_0^\infty\, f(k_1' - k_1) R' e^{i(k_1'x - {E't\over \hbar})}\, dk_1'
\eearray
and similarly in region 2,
\[ \Psi_2(x,t) = \int_0^\infty\, f(k_1' - k_1) \psi_2(x) e^{-i{E't\over \hbar}}\, dk' = \int_0^\infty\, f(k_1' - k_1) S' e^{-i)k_2'x + {E't\over\hbar}}\,dk_1'\]
where we have brought in the time dependence in order to have a traveling wave packet. 

The integration from $0$ to $\infty$ implies that we have a contribution of all plane waves with $E>V_1$. If $f(k_1'-k_1)$ is sharply peaked, then only those plane waves around $k_1'=k_1$ will give an significant contribution. 
In the 2nd expression, $\Psi_2$, we will have to be more restricted on the $f(k_1'-k_1)$ in that we want $E<V_2$. Otherwise we will be including plane wave solutions in our wave packet which do not satisfy the wave equation. So 
we will assume that we are well above the barrier, i., $E$ well above $V_2$ and that $f(k_1'-k_1)$ is sufficiently ``localized'' around $k_1$ so that the plane wave solutions corresponding to $E<V-2$ to not contribute to the wave packet in 
regioin 2. To treat the case with $E\ge V_2$ would be more complicated and I don't want to do it here.

Now let us look at the behavior of out constructed wave packets. The 1st term of $\Psi_1$ is $\displaystyle \int_0^\infty\, f(k_1' - k_1) e^{-i(k_1'x + {E't\over\hbar})}\, dk_1'$ and represents a wave packet traveling to the left with a center at 
$x=-v_1t$ and with a ``group'' velocity of $\displaystyle v_1 = {d\omega\over dk} = +{1\over\hbar}\left.{dE'\over dk_1'}\right|_{k_1' = k_1}$. (See pages 5 + 6 Section IV where we discussed the velocity of the center of a wave packet).

We have $\displaystyle E-V_1 = {p_1^2\over2m} = {\hbar^2k^2\over2m}$ and so $\displaystyle v_1 = {\hbar k_1\over m} = {p\over m}$. So in this case, the center of the wave packet moves with the classical velocity. Since $x=-v_1t$, we see that in the remove past, the center of the wave packet was far to the right and moved with a constant velocity to the left reaching the origin at time $t=0$.

Similarly, the 2nd term in $\Psi_1$ represents a wave packet \underline{moving to the right} with the same velocity $v_1$ with its center at $x=v_1t$. So this ``reflected'' packet starts out at the origine at time $t=0$ and moves with a constant 
velocity to the right.

We can ask what in the above precludes the possibility that this ``reflected'' packet was already moving to the right at a large distance in the remote past. i.e., long before out incident packet, the 1st term in $\Psi_1$, could reach 
the potential step at $x=0$ and be reflected. In the 2nd term we have a phase in the exponent of $\displaystyle \phi' = k_1' - {E't\over\hbar}$. For large $x$ and $t\ll0$ (a large negative $t$ to the remote past), $\phi'$ is a positive quantity with 
$\displaystyle e^{i(k_1'x - {E't\over\hbar})}$ oscillating rapidly as we integrate over $k_1'$. Consequently there will be essentially no contribution to the integral in this case. This $\Psi_1$(2nd term) will only be ``large'' when the phase of 
$\phi'$ is statioinary i.e., $x$ and $t$ both positive with $x = v_1t$. Thus the reflected packet does not ``appear'' until the incident packet arrives and is reflected. 

Similar arguments can be made for the incidence packet $\Psi_1$ (1st term), which has a phase $\displaystyle \phi' = k_1' + {E't\over\hbar}$. This packet will only exist for $x>0$ and $t<0$ with $x = -v_1t$. At $t=0$, when it reaches the potential step, it ``disappears'' never to be seen again.

Using the same kind of arguments, $\Psi_2$ represents a wave packet which ``originates'' at the origin at time $t=0$ and travels to the left. As $\displaystyle  \phi' = k_2' x + {E't\over\hbar}$, the phase is stationary at 
$$\left.{dk_2'\over dk_1'}\right|_{k_1' = k_1'} x + \left.{dE'\over dk_1'}\right|_{k'=k} {t\over\hbar} = {dk_2\over dk_1} x + v_1 t = 0.$$
As $k_1^2 = k_2^2 + U_2 - U_1$, $\displaystyle {dk_2\over dk_1} = {k_1\over k_2} = {v_1\over v_2}$. 
and we see that the center of our ``transmitted'' packet is at $x = -v_2t$ traveling with its classical velocity to the left.

Note that the reflected packet is reduced by $R$ from the incident packet and the transmitted packet is reduced by $S$. The only difference between this approach and a classical picture is that a classical particle 
would not have been reflected but would have continued on into region 2 with a decreased velocity of $v_2$. Light waves on the other hand would be ``reflected'' and ``transmitted'' when they struck an interface 
between two dielectric mediums.

\underline{Case 2: $U_1 < \E < U_2$} Now we look at the other case where the energy is below $V_2$. Things will go much the same here as in Case 1 but with some important differences. 

The plane was solutions can be written as (see Messiah pp 82 for example)
$$\left.\begin{matrix}
\psi_1 &=& e^{-ik_1x} - e^{ik_1x + 2\phi} \\
\psi_2 &=& A e^{i\phi}e^{k_2x}\\
&\null &\mbox{ decaying exponential as } x \mbox{ is negative} 
\end{matrix}\right\}
\mbox{ where }
\left\{\begin{matrix}
k_1 &=& \sqrt{\E - U_1} \\
k_2 &=& \sqrt{U_2 - \E} \\
A  &\null&\mbox{ is a constant}\\
\sin\phi &=& \sqrt{{\E-U_1\over U_2 - U_1}}  = {k_1\over \sqrt{k_1^2 + k_2^2}} 
\end{matrix}\right. $$

Using $f(k_1' - k_1)$, we construct the packets
\bearray
 \Psi_1(x,t) &=&  \int_0^\infty\, f(k_1' - k_1) e^{-(k_1'x + {E't\over \hbar})}\, dk_1' + \int_0^\infty\, f(k_1' - k_1) e^{i(k_1'x + 2\phi' - {E't\over \hbar})}\, dk_1'  \mbox{ ~~and }\\
 \Psi_2(x,t) &=&  A e^{k_2x} \int_0^\infty\, f(k_1' - k_1) e^{-(k_1'x + {E't\over \hbar})}\, dk_1''\\
 \eearray using a real $f(k_1' - k_1)$ centered around $k_1' = k_1$. 
 
 By similar arguments to those given on 8-IV, $f(k_1'-k_1)$ should be zero for $k_1'$'s corresponding to $E> V_2$ and we will stay away from the top of the barrier where $E\le V_2$.
 
 As in Case 1 above, the 1st term in $\Psi_1$ represents a wave packet traveling to the left with the classical velocity $v_1$, centered at $x = - v_1t$, reaching the origin at $t=0$, and ``disappearing''.
 
 Similarly, the 2nd term in $\Psi_1$ represents a packet, non-existent for $t<0$, traveling to the right with the same velocity, $v_1$, which as its center at 
 $$ x = v_1t - \left.2{d\phi'\over dk_1'}\right|_{k_1' = k_1}.$$
 
We note that the center of the incident packet reaches the origin at $t=0$ while the center of the reflected packed doesn't leave the origin until a time $\displaystyle t = {2\over v_1}{d\phi\over dk_1}$ later. now how significant is this
``delay'' time and can we gain some insight into its meaning? The answer is both Yes! and No!

If we want to compute the motion of our packet with a classical particle, then it should retain a nice localized shape -- at least away from the origin where the potential step is. Near the origin we can expect
some distortion as the ``front part'' of the packet arrives first and is reflected before the ``back part'' of the packet arrives. In order to have a nice localize packet, we don't want the $2\phi'$ term in the phase varying too 
rapidly with $k_1'$ over the range $\Delta k$ where $f(k_1' - k_1)$ is large. That is, we want 
$$\Delta k \left(2 {d\phi\over dk_1}\right) \ll 1.$$ or using $\Delta x \Delta k \approx 1$, we want $\displaystyle \Delta x \gg 2{d\phi\over dk_1} = v_1 \tau$.  This last relationship tells us that the delay time, $\tau$, is much
less than the time it takes the packet to move its mean width $\Delta x$. Since we are considering a bundle of waves, it is not reasonable to ask details about what goes on during such a short time. 

But we can gain some insight into what is going on by looking at $\Psi_1(x,t)|$. $|\Psi_2|$ will only be large when $\displaystyle \left[{d\over dk_1;}(\phi' - {E't\over \hbar})\right] \approx 0$. That is, when 
$\displaystyle {d\phi\over dk_1} = {t\over\hbar}{dE\over dk_1}$ which is when $t=\tau/2$. In constrast to a classical particle which never gets into region 2. the packet has a good chance of existing in region 2 near the
origin for a time of order $\tau/2$. How far it gets into region 2 depends on the size of $k_2$. So this is purely a quantum mechanical result where the particle penetrates into region 2, finds out that it ``shouldn't be there''
and ``gets out''.

Could we ever detect the particle in there? One can also ask similar question such as can a baseball go through a glass window without breaking it?

\subsection{Spreading of the Wave Packets}
Let us now see how wave packets in general will spread. Qualitatively we expect this to happen as a packet is made up of several momenta and one would expect the faster components to ``get ahead'' of the slower ones. 

Now a particle in a potential which is independent of its momentum, viz. $$E = \hbar\omega = {p^2\over2m} + v(x),$$ has a group velocity of $\displaystyle v_g = {\partial \omega\over \partial k} = {\hbar k \over m} = {p\over m} = v_cl.$
we see that the center of this packet moves with a constant velocity, the classical velocity, $v_cl$, but the packet will still spread as we shall see below.

Dropping the prims in the integration, we consider a packet with a Gaussian shaped weighting function, $\displaystyle f(k-k_0) ={1\over\sqrt{2\pi}}e^{-{(k-k_0)^2\over 2 \Delta k^2}}$, which is centered around $k = k_0$ with a width $\Delta k$ defined by $\Delta k^2 = \left[ \expect{k^2} - \expect{k}^2\right]_{t=0}$. Furthermore, let's assume that the packet is centered at $x_0$ at time $t_0$.

Then our packet is given by 
$$\Psi(x,t) = {1\over\sqrt{2\pi} } \int_{-\infty}^\infty \, e^{-{( k-k_0)^2\over2\Delta k^2}}\, e^{i[k(x-x_0) - \omega t]}\, dk.$$
Now we can't do the integration until we know how $\omega$ varys with $k$. If it was a free particle, $\displaystyle E = \hbar\omega = {p^2\over2m} = {\hbar^2k^2\over2m}$ or if the particle moved
in a momentum independent potential, $\displaystyle E = \hbar\omega = {\hbar^2k^2\over2m} - V(x)$, then we would know how $\omega$ depended on $k$ and we can go ahead an integrate. 

Even if we didn't know how $\omega$ depended on $k$ but if $\Delta k$ was small enough, we could hope to expand $\omega$ about $k = k_0$ and neglect the higher order terms as being
small. That is, the higher order terms will be small for small $k-k_0$ and will be ``cut off'' by $f(k-k_0)$ for large $(k-k_0$. so 
$$\omega(k) = \omega(k_0) + (k-k_0) \left.{\partial\omega\over\partial k}\right|_{k=k_0} + {1\over2}(k-k_0)^2\left.{\partial^2\omega\over\partial k^2}\right|_{k=k_0} + \hdots.$$
Let use just keep the 1st three terms and define $\displaystyle K = (k-k_0), X = x-x_0, \omega_0 = \omega(k_0), v_g =\left.{\partial\omega\over\partial k^2}\right|_{k=k_0}$ and
$\displaystyle \alpha = \left.{\partial^2\omega\over\partial k^2}\right|_{k=k_0}.$

With all this, our wave packet become
$$\Psi(x,t) = {1\over\sqrt{2\pi}}\int_{-\infty}^\infty\, e^{ {K^2\over2\Delta k^2}} e^{i\left[ Kx - v_gK t - {1\over2}\alpha K^2t\right]} e^{i[k_0X - \omega4]} \, dK.$$
We can complete the square in the exponents of the 1st two terms obtaining an integral of the form
$$\int_{-\infty}^{\infty}\, e^{-a^2y^2}\, dy = { \sqrt{\pi}\over a}.$$
This procedure give us 
$$\Psi(x,t) = \sqrt{ {\Delta k^2\over 1 + i\alpha\Delta k^2 t}} \, e^{-\left[ {(x-v_gt)^2\Delta k^2\over 2(1+ \Delta k^2 \alpha^2y^2}\right]} \, e^{i\left[ {\alpha\Delta k^4+(x - v_gt)^2\over 2(1+\alpha^2\Delta k^4 t^2} + k+0x=\omega_0 t\right]}$$
from which $$ |\Psi(x,t)|^2 = \left| {\Delta k^2\over (1+ i\alpha\Delta k^2 t)}\right| \, e^{-\left[{\Delta k^2(x-v_gt)^2\over 1 + \alpha^2\Delta k^2 t^2}\right]}.$$
If $\alpha = 0$, we see that we will have a wave packet of constant shape centered around $x = x_0 + v_g t$. But if $\alpha\ne 0$, we see that the amplitude of the packet will decrease in time and that the packet will also ``spread'' in time. 
A measure of this spreading can be taken as the width of the packet at $1/e$ of its maximum value, i.e., $$\Delta x\approx {\sqrt{1 + \alpha^2\Delta k^t t^2\over \Delta k}}.$$

Let $\displaystyle \left. \delta x\right|_{t=0} = \Delta x \approx {1\over \Delta k}$. Then we have 
$$\delta x = \Delta x\sqrt{ 1 + {\alpha^2t^2\over \Delta x^4}}.$$
We see for small times, where $\displaystyle {\alpha^2 t^2\over \Delta x^4} \ll 1$, that $\delta x \approx \Delta x$ and the packet hasn't spread much. For much longer times, where $\displaystyle {\alpha^2t^2\over \Delta x^4}\gg 1$, we have
$\displaystyle \delta x = {\alpha t \over \Delta x}$ and the spreading increases linearly with $t$.

For a free particle, $\displaystyle \left(E = {p^2\over2m}\right)$, or for a particle moving in a potential, $\displaystyle E = {p^2\over2m} + V(x)$, we have $\displaystyle \alpha = \left. {\partial \omega\over\partial k}\right|_{k=k_0} = {\hbar\over m}$. So all packets of interest will spread as they move along. For classical particles, it doesn't affect us because $\displaystyle {\hbar\over m}$ is very small. 